\documentclass{article}
\usepackage{amsmath,amssymb,amsthm,ulem,bm}
\usepackage[margin=1 in]{geometry}
\author{2019011985\and M91\and Junzhe Dong}
\title{Homework 7 for Measure and Integral}
\begin{document}
\maketitle
\newcommand{\st}{\text{ s.t.}}
\newcommand{\dd}{\,\mathrm{d}}
\newcommand{\du}{\,\mathrm{d} \mu}
\newcommand{\re}{\mathrm{Re}\,}
\newcommand{\im}{\mathrm{Im}\,}
\newcommand{\sip}{\mathrm{Simp}}
\newcommand{\R}{\mathbf{R}^d}
\newcommand{\WLOG}{without loss of generality}
\newcommand{\aeeq}{\stackrel{\mathrm{a.e.}}{=}}
\newcommand{\B}{\mathcal{B}}
\newcommand{\F}{\mathcal{F}}
\newcommand{\Leb}{\mathcal{L}}

\DeclareRobustCommand{\rchi}{{\mathpalette\irchi\relax}}
\newcommand{\irchi}[2]{\raisebox{\depth}{$#1\chi$}} 

\paragraph{1.4.40}
\subparagraph{(ii)}
\begin{proof}
$\forall c\in\mathbf{C},f,g\in L(X,\B,\mu)$:

\textbf{``$\bm{+}$'':}It suffice to assume $f,g$ real, since by definition the integral is the linear combination of its real and imaginary parts. 

First assume $g>0$, then by linearity of unsigned integral:
\[\int_{X}(f+g)\dd\mu=\int_{X}f_++g\dd\mu-\int_{X}f_-\dd\mu=\int_X f\dd\mu+\int_X g\dd\mu\]
Similarly, the equation holds for $g\leq 0$.

Pointwisely, $(f+g)(x)=(f+g)_+(x)+(f+g)_-(x)=f_+(x)+f_-(x)+g_+(x)+g_-(x)$, so
\[\begin{aligned}
\int_X f+g\dd\mu&=\int_X f_++g_+-f_-g_-\dd\mu\\
&=\int_X f_++g_+-f_-\dd\mu-\int_X g_-\dd\mu\\
&=\cdots\\
&=\int_X f_+\dd\mu+\int_X g_+\dd\mu-\int_X f_-\dd\mu-\int_X g_-\dd\mu\\
&=\int_{X}f\dd\mu+\int_{X}g\dd\mu
\end{aligned}\]

\textbf{``$\bm{\times}$'':}Suppose $c=a+ib, f(x)=u(x)+iv(x), \text{ where } a,b,u(x),v(x)$ are real. Then 
\[\begin{aligned}
&\int_X cf\dd\mu\\=&a\int_X u\dd\mu-b\int_X v\dd\mu+ia\int_X v\dd\mu+ib\int_{X}u\dd\mu\\
=&(a+bi)\int_{X}u+iv\dd\mu=c\int_X f\dd\mu
\end{aligned}\]
\end{proof}
\subparagraph{(i)}
\begin{proof}
``+'': Associativity, commucativity are trivial. Closedness follows from $\|f+g\|\leq\|f\|+\|g\|<\infty$. +identity 0$\in \Leb(X,\B,\mu)$, with $-f$ serving as the inverse of $f$, which follows from the liniearity of unisigned integral and definition.

``$\cdot$'': It direcly follows from (ii).
\end{proof}

\subparagraph{(iii)}
\begin{proof}
\textbf{triangle inequality:} Now that we have the pointwise triangle inequality: $|f(x)+g(x)|\leq |f(x)|+|g(x)|$, from linearity properties, we see that $\|f+g\|_{L^1(\mu)}=\int_X |f+g|\dd\mu\leq\int_X|f|+|g|\dd\mu\leq \|f\|_{L^1(\mu)}+\|g\|_{L^1(\mu)}$.

\textbf{homogeneity property:} Denote $c=re^{i\theta}, r=|c|$. Then from liniear properties of unsigned functions: $\|cf\|_{L^1(\mu)}=\int_X |e^{i\theta}rf|\dd\mu=\int_X |rf|\dd\mu=r\int_X |f|\dd\mu=|c|\|f\|_{L^1(\mu)}$
\end{proof}
\subparagraph{(iv)}
\begin{proof}
By exercise 1.4.35, the equation holds for unsigned measurable functions. Since $\int_X f \dd\mu=[\int_X\re(f)_+\dd\mu-int_X\re(f)_-\dd\mu]+i[int_X\im(f)_-\dd\mu+int_X\im(f)_-\dd\mu]$, and the euality holds for each component of the equation, the statement follows.
\end{proof}
\subparagraph{(v)}
\begin{proof}
\textbf{``$\bm{\leq}$'':}By linearity, it suffice to show the case for unsigned absolutely measurable functions. On one hand, $\forall g\in \sip^+(X,\B,\mu),g\in \sip^+(X,\B',\mu')$. So 
\[\int_{X}f\dd\mu=\sup\limits_{\substack{g\leq f\\g\in\sip^+(X,\B,\mu)}}\sip\int_{X}g\dd\mu\leq\sup\limits_{\substack{g\leq f\\g\in\sip^+(X,\B',\mu')}}\sip\int_{X}g\dd\mu'=\int_{X}f\dd\mu'\]

\textbf{``$\bm{\geq}$'':} By verticle truncation and horizontal truncation, WLOG, we may assume that $f$ is bounded and has bounded support. Then it suffice to show that $\forall g\in \sip^+(X,\B',\mu'), g\leq f, \int_{X}f\dd\mu\geq \sip\int_{X}f\dd\mu'$. Denote $g(x)=\sum\limits_{n=1}^{N}c_n1_{A_n}$, where $\forall i\neq j, A_i\cap A_j=\varnothing, A_i\in\B'$. Then $\forall\varepsilon>0, \exists f_\varepsilon\in\sip^+(X,\B,\mu),f_\varepsilon\leq f\st\forall x\in A_i, f_\varepsilon(x)>(1-\varepsilon)c_i$. Then $\int_{X}f\dd\mu\geq \sip\int_{X}f_\varepsilon\dd\mu\geq (1-\varepsilon)\sip\int_{X}g\dd\mu$. Take $\varepsilon\to 0$ and the desired inequality follows.
\end{proof}
\subparagraph{(vi)}
\begin{proof}
The ``if'' side is trivial. For the ``only if'' side, suppose the converse is true, then $\exists c>0\st \mu(\{\left|f(x)\right|>c\})>0$. By Markov's inequality, we learn that $\|f\|>c\mu(\{|f(x)|>c\})>0$, which is in denial of the assumption.
\end{proof}
\subparagraph{(vii)}
\begin{proof}
$\forall O\subset \mathbf{C}$ open, $f\downharpoonright_{Y}^{-1}(O)=f^{-1}(O)\cap Y\subset \B\downharpoonright_{Y}$, so $f\downharpoonright_{Y}$ is indeed measurable. 

As was proved in previous homework, the equality holds for unsigned simple function. From liniearity, it suffice to prove the equality for unsigned measurable functions. Take $f_n\in\sip^+(X,\B,\mu)\st f_n\leq f1_Y\in\Leb^1(X,\B,\mu),\left|\int_X f_n\dd\mu-\int_X f1_Y\dd\mu\right|=\int_X(f-f_n)1_Y+\int_X f_n1_{Y^c}<\frac{1}{n}\Rightarrow \int_X(f-f_n)1_Y<\frac{1}{n}$. Since $\int_Y f_n\dd\mu=\int_{X}f_n1_Y\dd\mu$, $\int_X f1_Y\dd\mu=\lim\limits_{n\to\infty}\int_Y f_n\dd\mu\leq \int_Y f\dd\mu$. Meanwhile, $f\downharpoonright_Y(x)\leq f(x)1_Y,\forall x\in Y$, and that $f$ is unsigned, so $\int_{Y}f\dd\mu\leq \int_{X}f1_{Y}\dd\mu$. Combine the two inequalities and we get the desired equation. 

%Since $\sip\int_{Y}f_n\dd\mu=\sip\int_Y f_n1_{Y}\dd\mu, f_n1_{Y}\leq f\downharpoonright_{Y}$, so $\int_Y f\dd\mu\geq $ 
\end{proof}

\paragraph{1.4.42}
\begin{proof}
Define $f_1(x)=1_{[0,1]}, f_n(x)=1_{[n-1,n]}-1_{[n-2,n-1]}\quad \forall n\geq 2$. Obviously, $\sum\limits_{n=1}^{\infty}f_n(x)$ absolutely converges, yet
\begin{gather*}
\int_{X} \sum_{n=1}^{\infty}f_n\dd\mu =\int_{X}0\dd\mu=0\\
\sum_{n=1}^{\infty}\int_{X}f_n\dd\mu=1
\end{gather*}
Which is a counter example.
\end{proof}

\paragraph{1.4.45}
\begin{proof}
Denote $g_n(x)=f_n(x)-f(x)$, $|g(x)|\leq |f(x)|+|f_n(x)|\leq 2G(x)$, so $g_n(x)$ are dominated by $2G(x)$, which is still absolutely integrable. Apply dominated convergence theorem and we get
\[\lim_{n\to\infty}\int_{X}|g_(x)|\dd \mu=\int_{X}\lim_{n\to\infty}|g_n(x)|\dd\mu\]
While $LHS=\lim\limits_{n\to\infty}\left\|f_n-f\right\|_{L^1}$, $RHS=\int_{X}|\lim\limits_{n\to\infty}f_n-f|\dd\mu=0$. The statement thus follows. 
\end{proof}

\paragraph{1.4.47}
\begin{proof}
Denote $g_n(x)=\min_\{f_n(x),f(x)\}$. Then $|g(x)|\leq |f(x)|$, which is absolutely integrable. So we may apply dominated convergence theorem to $g_n(x)$:
\[\begin{aligned}
\lim_{n\to\infty}\int_{X}g_n\dd\mu&=\int_{X}\lim_{n\to\infty}g_n(x)\dd\mu\\
\lim_{n\to\infty}\int_{X}\min\{f_n,f\}\dd\mu&=\int_{X}f\dd\mu\\
\end{aligned}\]
Meanwhile, in the scope of absolutely integrable functions, we have:
\[\min\{f_n,f\}=\frac{f_n+f}{2}-\frac{|f_n-f|}{2}\]
By linearaity, we have
\[\lim_{n\to\infty}+\int_{X}f_n\dd\mu+\int_{X}f\dd\mu-\lim_{n\to\infty}\|f_n-f\|=\int_{X}f\dd\mu\]
The equation follows after some trivial simplication.
\end{proof}

\newcommand{\unito}{\rightrightarrows}

\paragraph{1.5.5}
\begin{proof}
Let's prove (ii) directly, which implies (i).

Denote $g_n(x)=f_n(x)-f(x)$, then it's equivalent to prove that $g_n\unito 0$. Since $\sum\limits \|g_n\|_{L^1{\mu}}<\infty, \forall \delta>0,k\in\mathbf{N},\exists N_k\in\mathbf{N}\st \sum\limits_{n=N_k}^{\infty}\|g_n\|<\delta*4^{-k}$. 

Denote $A_{n,k}=\{x\in X|g_n(x)\geq 2^{-k}\}$. By Markov's inequality, $\mu(A_{n,k})\leq 2^{-k}\|g_n\|_{L^1(\mu)}$ . Denote $A_k=\{x\in X|\sup\limits_{n\geq N_k}g_n\geq 2^{-k}\}$. Obviously, $A_k\subset\bigcup\limits_{n\geq N_k}^{\infty}A_{n,k}$, so $\mu(A_k)\leq \sum_{n=N_k}^{\infty}\delta \|g_n\|*2^{-k}=2^{-k}\delta$. Denote $A=\bigcup_{k=1}^{\infty}A_k$, then $\mu(A)\leq\sum_{k=1}^{\infty}\delta2^{-k}=\delta$. 

So while $\mu(A)$ can be as small as possible, $\forall x\in A^c$, $\sup\limits_{n>N_k}g_n\leq 2^{-k}$, so by definition $g_n\unito 0$ on $A^c$, which is the definition of almost uniformly convergence. 
\end{proof}

\paragraph{1.4.50}
\begin{proof}
By definition, $\forall n\in\mathbf{N},\exists\text{simple functions } g_n\geq f,h_n\leq f\st$ (Denote $\int_{X}f\dd\mu$ the upper and lower integral since they agree: it's not yet the real integral, merely a notation)
\[\int_{X}f\dd\mu-\frac{1}{2^n}\leq\sip\int_{X}h_n\dd\mu\leq\int_{X}f\dd\mu\leq\sip\int_{X}g_n\dd\mu\leq\int_{X}f\dd\mu+\frac{1}{2^n}\]
Obviously, $\lim\limits_{n\to\infty}\int_{X}g_n\dd\mu=\lim\limits_{n\to\infty}\int_{X}h_n\dd\mu=\int_{X}f\dd\mu$. Apply Fatou's lemma to $g_n, h_n$ and we get
\begin{gather*}
\int_{X}f\dd\mu\leq\int_{X}\varliminf_{n\to\infty}g_n\dd\mu\leq\int_{X}f\dd\mu\\
\int_{X}\varliminf_{n\to\infty}h_n\dd\mu\leq\int_{X}f\dd\mu
\end{gather*}
Select the subsequence $\st$ $\varliminf\limits_{n\to\infty}g_n=\lim\limits_{k\to\infty}g_{n_k},\varliminf\limits_{n\to\infty}h_n=\lim\limits_{k\to\infty}h_{n_k}$.

To prove the statement, it suffice to show that $\lim\limits_{k\to\infty}g_{n_k}-f=0$. Meanwhile, $0\leq g_{n_k}-f\leq g_{n_k}-h_{n_k}$, so it suffice to show that $\lim\limits_{k\to\infty}g_{n_k}-h_{n_k}=0$ pointwise. Meanwhile, we learn that $\sum\limits_{n_k=1}^{\infty}\|(g_{n_k}-h_{n_k})-0\|_{L^1}\leq 1<\infty$, and that $0$ is a measurable function, so by exercise 1.5.5, $\lim_{k\to\infty}g_{n_k}-h_{n_k}=0$ pointwise. Thus the statement is proved.
%not yet solved
\end{proof}

\paragraph{1.5.2}
(i) is trivial by definition. (ii)(v) are trivial by taking the null set $E=\varnothing$.
\subparagraph{(iii)}
\begin{proof}
Suppose $E$ is the null set outside which $f_n$ converge uniformly to $f$. $\forall \varepsilon>0,\mu(E)<\varepsilon, f_n\downharpoonright_{E^c}\unito f$, so by definition it converges almost uniformly.
\end{proof}
\subparagraph{(iv)}
\begin{proof}
If not, $\exists E\st \mu(E)=c>0, \varepsilon_0>0,\forall N\in\mathbf{N},\exists n>N\st |f_n-f|(x)>\varepsilon$. Meanwhile, by definition, $\exists F\st\mu(F)<\frac{c}{2}, |f_n(x)-f(x)|<\varepsilon\quad \forall x\in F^c$. Since $\mu(F)<\mu(E)$, $F^c\cap E\neq\varnothing$, which is in denial of the assumption.  
\end{proof}
\subparagraph{(vi)}
\begin{proof}
By Markov's inequality, $\forall\varepsilon$ fixed, we have $\lim\limits_{n\to\infty}\mu\{x\in X:|f_n-f|(x)>\varepsilon\}<\lim\limits_{n\to\infty}\frac{1}{\varepsilon}\|f_n-f\|_{L^1(\mu)}=0$, which proves the statement by definition.
\end{proof}
\subparagraph{(vii)}
\begin{proof}
Suppose the converse is true. Then $\exists \varepsilon_0>0\st \lim\limits_{n\to\infty}\mu(\{x\in X:|f_n-f|(x)>\varepsilon\})=c>0$. Meanwhile, by definition, $\exists F\st\mu(F)<\frac{c}{2}, |f_n(x)-f(x)|<\varepsilon\quad \forall x\in F^c$, so $\lim\limits_{n\to\infty}\mu(\{x\in X:|f_n-f|(x)>\varepsilon\})<\frac{c}{2}$, which is a contradiction. 
\end{proof}

\paragraph{1.5.1}

\subparagraph{(i)}
\begin{proof}
All of them follows from definition trivially.
%$f_n\to f$ pointwise a.e. $\Leftrightarrow \forall x\in X a.e. \forall\varepsilon>0,\exists N\in\mathbf{N}\st \forall n>N, |f_n-f|<\varepsilon\quad\Leftrightarrow |f_n-f|\to 0\quad(n\to\infty)$.
%The same inequality applies to the first 5 convergence in different domain. 
%$f_n\to f$ in measure $\Leftrightarrow \forall \varepsilon>0, \lim\limits_{n\to\infty}\mu(\{|f_n-f|>\varepsilon\})=0\quad\Leftrightarrow |f_n-f|\to 0$ in measure
\end{proof}
\subparagraph{(ii)}
\begin{proof}
The case for the first 5 convergence follows directly from their linearity properties.

\textbf{convergence in $\Leb^1$-norm:} 
\[\begin{aligned}
\lim\limits_{n\to\infty}\|(f_n+g_n)-(f+g)\|&=\lim_{n\to\infty}\int_{X}|(f_n+g_n)-(f+g)|\dd\mu\\
&\leq\lim_{n\to\infty}\int_{X}|f_n-f|\dd\mu+\lim_{n\to\infty}\int_{X}|g_n-g|\dd\mu=0
\end{aligned}\]
\[\lim_{n\to\infty}\|cf_n-cf\|=|c|\lim_{n\to\infty}\|f_n-f\|=0\]

\textbf{convergence in measure:} Take $f_n\to f,g_n\to g$ in measure, then $\forall \varepsilon>0, \lim\limits_{n\to\infty}\mu(\{|f_n-f|>\varepsilon\})=0, \lim\limits_{n\to\infty}\mu(\{|g_n-g|>\varepsilon\})=0$. Since $|(f_n+g_n)-(f+g)|\leq |f_n-f|+|g_n-g|$, $\lim\limits_{n\to\infty}\mu(\{|(f_n+g_n)-(f+g)|>\varepsilon\})\leq \lim\limits_{n\to\infty}\mu(\{|f_n-f|>\varepsilon\})+\lim\limits_{n\to\infty}\mu(\{|g_n-g|>\varepsilon\})=0$, so $f_n+g_n\to f+g$ in measure. 

$\forall\varepsilon>0, \{|cf_n-cf|>\varepsilon\}=\{|f_n-f|>\frac{\varepsilon}{|c|}\}$. Take $n\to\infty$ on both sides and we get $\lim\limits_{n\to\infty}\mu(\{|cf_n-cf|>\varepsilon\})=0$, so $cf_n\to cf$ in measure.
\end{proof}
\subparagraph{(iii)}
\begin{proof}
The first 5 are similar, and we prove the case for pointwise convergence for instance.

\textbf{pointwise convergence:}
\[0\leq\varliminf\limits_{n\to\infty}|g_n|(x)\leq\varlimsup\limits_{n\to\infty}|g_n|(x)\leq\lim\limits_{n\to\infty}f_n(x)=0\]
So by squeeze theorem, the equality follows.

\textbf{$\Leb^1$ convergence:}
By monotonicity:
\[0\leq\varliminf\limits_{n\to\infty}\|g_n\|_{L^1}\leq\varlimsup\limits_{n\to\infty}\|g_n\|_{L^1}\leq\lim\limits_{n\to\infty}\|f_n\|_{L^1}=0\]
So by squeeze theorem, the equality follows.

\textbf{convergence in measure:}$\forall\varepsilon>0$:
\[0\leq\varliminf_{n\to\infty}\mu(\{g_n>\varepsilon\})\leq\varlimsup_{n\to\infty}\mu(\{g_n>\varepsilon\})\leq \lim_{n\to\infty}\mu(\{f_n>\varepsilon\})=0\]
So by squeeze theorem, the equality follows.
\end{proof}


\paragraph{1.5.6}
\begin{proof}
Let $g=f_n-f$, then it's equivalent to prove that $g_n$ almost uniformly converges to 0.

Denote $A_{n,k}=\{x\in X| |g_n|\geq 2^{-k}\}$. Since $g_n$ converges to 0 in measure, $\forall k\in\mathbf{N}$ fixed, $\lim\limits_{n\to\infty}\mu(A_{n,k})=0$. So $\forall k\in \mathbf{N}\text{ fixed },\delta>0,\exists N_k\in\mathbf{N}\st \forall n>N_k, \mu(A_{n,k})<\delta4^{-k}$.

Select the sub-sequence $\{h_n\}_{n=1}^{\infty}\subset\{g_n\}_{n=1}^{\infty}\st h_k=g_{N_k}$. Denote $A=\bigcup\limits_{k=1}^{\infty}A_{N_k.k}$, then $\mu(A)\leq\sum_{k=1}^{\infty}\mu(A_{N_k,k})=\delta$. So while the measure of $A$ can be as small as possible, $\forall x\in A^c, \sup\limits_{m>n}h_m(x)\leq 2^{-n}$. So by definition, $h_n\unito 0$ on $A^c$, which is the definition of almost uniform convergence.
%Then $A_m=\{x\in X|\sup\limits_{n>m}h_n\geq 2^{-k}\}$
\end{proof}

\paragraph{1.5.10}
\subparagraph{(i)}
\begin{proof}
\begin{enumerate}
\item Since $f$ is absolutely integrable, $\sup\limits_{n}\|f_n\|_{L^1(\mu)}=\int_{X}|f|\dd\mu<\infty$.
\item Denote $g^-_n(x)=|f|*1_{\{|f|\leq n\}}(x), g^+_n(x)=|f|*1_{\{|f_n|>M\}}$. So $\{g^-_n\}$ is a non-decreasing sequence of unisigned functions with $\lim\limits_{n\to\infty}g^-_n=|f|$. By monotone convergence theorem, we have:
\[\begin{aligned}
\int_{|f|>n}|f|\dd\mu &=\int_{X}g^+_n\dd\mu\\
&=\int_{X}f\dd\mu-\int_{X}g^-_n\dd\mu\\
\lim_{n\to\infty}\int_{|f|>n}|f|\dd\mu&=\int_{X}f\dd\mu-\lim_{n\to\infty}\int_{X}g^-_n\dd\mu\\
&=\int_{X}f\dd\mu-\int_{X}\lim_{n\to\infty}g^-_n\dd\mu\\
&=0
\end{aligned}\]
So $f_n$ does not escape to vertical infinity.
\item Denote $h^-_n(x)=|f|*1_{\{|f|\leq \frac 1 n\}}(x), h^+_n(x)=|f|*1_{\{|f_n|>\frac 1 n\}}$.  So $\{h^+_n\}$ is a non-decreasing sequence of unisigned functions with $\lim\limits_{n\to\infty}h^+_n=|f|$.By monotone convergence theorem, we have:
\[\begin{aligned}
\int_{|f|<\frac 1 n}|f|\dd\mu &=\int_{X}h^-_n\dd\mu\\
&=\int_{X}f\dd\mu-\int_{X}h^+_n\dd\mu\\
\lim_{n\to\infty}\int_{|f|<\frac 1 n}|f|\dd\mu &=\int_{X}f\dd\mu-\lim_{n\to\infty}\int_{X}h^+_n\dd\mu\\
&=\int_{X}f\dd\mu-\int_{X}\lim_{n\to\infty}h^+_n\dd\mu\\
&=0
\end{aligned}\]
So $f_n$ does not escape to width infinity.
\end{enumerate}
\end{proof}
\subparagraph{(ii)}
\begin{proof}
Suppose the sequence of measurable functions $\{f_n(x)\}$ is dominate by the absolutely integrable function $G(x)$
\begin{enumerate}
\item $\sup\limits_{n}\|f_n\|_{L^1(\mu)}\leq \|G\|<\infty$
\item $\sup_{n}\int_{|f_n|\geq M}|f_n|\dd\mu\leq \int_{|G|\geq M}G\dd\mu\to 0\quad (M\to \infty)$ by (i).
\item $\sup_{n}\int_{|f_n|\leq \delta}|f_n|\dd\mu\leq \int_{|G|\leq \delta}G\dd\mu\to 0\quad (\delta\to 0)$ by (i).
\end{enumerate}
\end{proof}
\subparagraph{(iii)}
$f_n(x)=1_{[n,n+1]}(x)$
\paragraph{1.5.17}
\begin{proof}
WLOG, assume that $f_n$ converges to $f$ pointwise since adjusting the value on a null set does not influence the integral.

\textbf{``if'' side:} Since $f_n,f$ are bounded, it's legitimate to denote $g_n=f_n-f$. Then $|g|\leq |f|+|f_n|<\infty,\sup\limits_{n}\int_{X}g_n\dd\mu<\infty$, $g_n(x)\to 0\quad(n\to\infty)$ pointwise. By assumption, $\forall \varepsilon>0,\exists N\in\mathbf{N}\st \forall n>N, \left|\int_{X}g_n\dd\mu\right|<\varepsilon$ and we're required to prove that $\int_{X}|g_n|\dd\mu\to 0\quad(n\to\infty)$. Denote $g_n^+, g_n^-$ the positive and negative parts of it as was defined in the definition, then $\left|\int_{X}g_n\dd\mu\right|=\left|\int_{X}g_n^+\dd\mu-\int_{X}g_n^-\dd\mu\right|,\int_{X}\left|g_n\right|\dd\mu=\int_{X}g_n^+\dd\mu+\int_{X}g_n^-\dd\mu$. So by linearity, it suffice to prove that $\int_{X}g_n^+\dd\mu\to 0\quad(n\to\infty)$. This is obvious since $g_+$ is dominated by $|f|+|f_n|$, and that $g\to 0$ pointwise.
%A bad proof
\textbf{``only if'' side:} If $f_n$ converges in $\Leb^1$ norm, then $\forall\varepsilon>0,\exists N\in\mathbf{N}\st \forall n>N, \int_{X}\left|f_n-f\right|\dd\mu<\varepsilon$. Meanwhile, by the triangle inequality, $\left|\int_{X}f_n-f\dd\mu\right|\leq \int_{X}\left|f_n-f\right|\dd\mu<\varepsilon$, so $\int_{X}f_n\dd\mu$ converges to $\int_{X}f\dd\mu$.
\end{proof}

\paragraph{1.5.19}
\subparagraph{(i)}
Refer to exercise 1.5.2: from (i)(v), we can redeuce the study of uniform convergence, pointwise convergence to thhat of pointwise a.e. convergence; from (iii)(iv), we can reduce the study of convergence in $\Leb^{\infty}$-norm, almost uniformly to that of pointwise a.e. convergence. From (vi), we may reduce the study of convergence of $\Leb^1$ convergence to that of measure convergence. So it suffice to study pointwise a.e. convergence and 
convergence in measure.
\begin{proof}
Notice that $|F_n(\lambda)-F(\lambda)|=|\mu(\{f(x)\leq\lambda\})-\mu\{f_n(x)\leq\lambda\}|=|\mu(\{f(x)\leq\lambda\}\cap\{f_n(x)>\lambda\})-\mu(\{f(x)>\lambda\}\cap\{f_n(x)<\lambda\})|$.

\textbf{pointwise a.e. convergence:} Since adjusting the value on a null set does not influnce $\mu(\{f_n(x)\leq\lambda\})$, so we may assume that $f_n\to f$ pointwise.

Since $f_n\to f$ pointwise, $\forall y_0=f(x_0)\st y_0<\lambda, \exists N_{x_0}\in \mathbf{N}\st \forall n>N_{x_0}, |f_n(x_0)-y_0|<\lambda-y_0$. Similar properties hold for $y_0>\lambda$. So  $\lim\limits_{n\to\infty}|F(\lambda)-F_n(\lambda)|=\lim\limits_{n\to\infty}\mu(\{f_n>\lambda\}\cap\{f=\lambda\})=0$ where $F$ is continuous ($\st \lim\limits_{\varepsilon\to 0}F(\lambda+\varepsilon)-F(\lambda)=\lim\limits_{n\to\infty}\mu(\{f=\lambda\})=0$).

\textbf{convergence in measure:} $\forall y_0=f(x_0)>\lambda, \lim\limits_{n\to\infty}\mu(\{f_n-y_0\}<|y_0-\lambda|)=0$. Similar properties hold for $y_0=f(x_0)<\lambda$. So $0\leq\lim\limits_{n\to\infty}|F_n(\lambda)-F(\lambda)|\leq\lim\limits_{n\to\infty}\mu(\{f=\lambda\}\cap\{f_n>\lambda\|)\leq\mu(f=\lambda)=0$ where $F$ is continuous at $\lambda$. By squeeze theorem, the statement follows.
\end{proof}
\subparagraph{(ii)}
$f(x)=1_{[0,\frac{1}{2}]}$, $f_n(x)=(1-\frac 1 n)1_{[\frac 1 2,1]}$. Then 
\[F(\lambda)(f)=\begin{cases} \frac 1 2 &\lambda<1\\1&\lambda=1\end{cases}\]
\[F(\lambda)(f_n)=\begin{cases} \frac 1 2 &\lambda<1-\frac 1 n\\1&\lambda\geq 1-\frac 1 n\end{cases}\]
By definition, $f_n\to f$ in distribution.
\subparagraph{(iv)}
$g(x)=1_{[\frac 1 2,1]}$. Consider $f,f_n$ in (ii), where $f_n\to g$ pointwise, thus in distribution. Obviously, $\mu(\{f\neq g\})\neq 0$.
\subparagraph{(iii)}
Take $f_n, f$ in $(ii)$, $h_n\equiv 0, h=0$. As was claimed in $(iv)$, $f_n+h_n\to g\neq f+0$ in distribution.



\end{document}