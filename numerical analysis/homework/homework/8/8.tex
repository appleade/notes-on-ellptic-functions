\documentclass{ctexart}
\usepackage{amsmath,amssymb,amsthm,bm,ulem,graphicx, booktabs}
\usepackage[margin=1 in]{geometry}
\title{数值分析作业8}
\author{数91\and 董浚哲\and 2019011985}
\begin{document}
\maketitle
\newcommand{\R}{\mathbf{R}}
\newcommand{\dd}{\,\mathrm{d}}
\newcommand{\st}{\text{ s.t. }}
\newcommand{\pp}[2]{\frac{\partial #1}{\partial #2}}
\newcommand{\nm}[1]{\left\|#1\right\|}

\paragraph{1.}
\begin{proof}
若$x_0>0$,则$\varphi(x_0)=x_0+x_0^3>x_0$;若$x_0<0$,则$\varphi(x_0)=x_0+x_0^3<x_0$。综上,$\nm{\varphi(x_0)-0}>\nm{x_0-0}$。同理,知$\forall k\in\mathbf{N}, \nm{\varphi(x_k)-0}>\nm{x_k-0}$。故迭代不收敛。

记$x^{(1)}=\varphi(x),x^{(2)}=\varphi\circ\varphi(x)$,则$\psi(x)=x-\frac{(x^{(1)}-x)^2}{x^{(2)}-2x^{(1)}+x}=x(1-\frac{1}{x^4+3x^2+2})\Rightarrow \psi'(x_k)=1+3\frac{x^4+x^2-1}{(x^4+3x^2+3)^2}$。故当$|x|<\sqrt{\frac{1+\sqrt{5}}{2}}$时,$|\psi'(x)|<1$,迭代收敛。故Steffensen迭代法局部收敛。
\end{proof}

\paragraph{2.}
\begin{proof}
$\varphi(x)=x-\frac{x^3-a}{3x^2}\Rightarrow \varphi'(x)=\frac{2}{3}[1-\frac{a}{x^3}]$。故$\forall a>0,x>0,|\varphi'(x)|<1$,故迭代法收敛。
\end{proof}

\paragraph{3.}
$\varphi(x)=px+q\frac{a}{x^2}+r\frac{a^2}{x^5}$。因迭代法收敛至$a^{\frac{1}{3}}$,故$\varphi(a^{\frac{1}{3}})=a^{\frac{1}{3}}\Rightarrow p+q+r=1$。为使收敛的阶尽可能高,而三个未知数最多由3个线性无关的方程完全决定,故需要
\[\varphi'(a^{\frac{1}{3}})=0\Rightarrow p-2q-5r=0\]
\[\varphi''(a^{\frac{1}{3}})=0\Rightarrow q+5r=0\]
故$(p,q,r)$满足方程
\[\begin{bmatrix}1&1&1\\1&-2&-5\\0&1&5\end{bmatrix}\begin{bmatrix}p\\q\\r\end{bmatrix}=\begin{bmatrix}1\\0\\0\end{bmatrix}\]
解之,得$p=q=-\frac{5}{9},r=-\frac{1}{9}$

\paragraph{4.}
\subparagraph{(1)}
\begin{proof}
$E_{k+1}=I-AX_{k+1}=I-AX_k(2I-A_k)=I-2AX_k+AX_kAX_k=(I-AX_k)^2=E_k^2$
\end{proof}
\subparagraph{(2)}
\begin{proof}
条件即$\nm{E_0}<1$。因$E_{k+1}=E_k^2, \nm{E_{k+1}}=\nm{E_k^2}\leq \nm{E_k}^2\Rightarrow \nm{E_k}\leq \nm{E_0}^{2k}$。故知$\lim\limits_{k\to\infty}\nm{E_k}=0\Rightarrow \lim\limits_{k\to\infty}I-AX_k=0\Rightarrow \lim\limits_{k\to\infty}X_k=A^{-1}$,得证。
\end{proof}
\subparagraph{(3)}
\begin{proof}
$e_k=A^{-1}(I-AX_k)=A^{-1}E_k\Rightarrow \nm{e_{k+1}}\leq \nm{A}^{-1}\nm{E_k}^2\leq \nm{A}^{-1}\nm{A}^2\nm{e_k}^2=\nm{A}\nm{e_k}^2$,故由定义至少二阶收敛。
\end{proof}

\paragraph{5.}
记$\Delta A=(a_{ij}),A^k=(a_{ij}^k)=,p^k=(p_i),q^k=(q_i)$。构造Lagrange函数为
\[f(A,\lambda)=\frac{1}{2}\nm{\Delta A}_F^2-\lambda^T(\Delta Ap^k+A_kp^k-q^k)\]
求偏导得
\[\pp{f}{a_{ij}}=a_{ij}-\lambda_i p_j=0\]
\[\pp{f}{\lambda_i}=-\sum_{j=1}^n (a_{ij}+a^k_{ij})p_j+q_i=0\]
也即
\[\Delta A=\lambda (p^k)^T\]
\[(\Delta A+A^k)p^k=q^k\]
在后式两端同乘$\lambda^T$,并代入前式整理得
\[\lambda(p^k)^Tp^k=q^k-A^kp^k\]
整理得$\lambda=\frac{q^k-A^kp^k}{\nm{p^k}_2^2}$,带回原式即得$\Delta A=\frac{(q^k-A^kp^k)(p^k)^T}{\nm{p^k}_2^2}$,得证。
\end{document}