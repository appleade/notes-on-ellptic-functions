\documentclass{ctexart}
\usepackage{amsmath,amssymb,amsthm,bm,ulem}
\usepackage[margin=1 in]{geometry}
\title{数值分析作业1}
\author{数91\and 董浚哲\and 2019011985}
\begin{document}
\maketitle
\newcommand{\R}{\mathbf{R}}
\newcommand{\dd}{\,\mathrm{d}}
\newcommand{\st}{\text{ s.t. }}
\newcommand{\pp}[2]{\frac{\partial #1}{\partial #2}}
\newcommand{\nm}[1]{\left\|#1\right\|}

\paragraph{1.}
\begin{proof}
设$x=(x_1,\cdots, x_n)$.

一方面,
\[\nm{A}_\infty=\max_{\nm{x}_\infty=1}\nm{Ax}_\infty=\max_{1\leq i\leq n}\{\sum_{j=1}^na_{ij}x_j\}\leq \max_{1\leq i\leq n}\{\sum_{j=1}^n |a_{ij}|\}\]
\end{proof}
故$\nm{A}_\infty\leq \max\limits_{1\leq i\leq n}\{\sum\limits_{j=1}^n |a_{ij}|\}$。

另一方面,对于上述$i$,$x_j=\mathrm{sign}(a_{ij})$时,$\nm{Ax}_\infty=\max\limits_{1\leq i\leq n}\{\sum_{j=1}^n |a_{ij}|\}$,故由定义$\nm{A}_\infty\geq =\max\limits_{1\leq i\leq n}\{\sum_{j=1}^n |a_{ij}|\}$

综上两方面即得所需结论。

\paragraph{2.}
\begin{proof}
考察方程$Ax=b$,$\nm{A}_\infty$即在$\nm{b}_\infty=1$的限制下$\nm{x}_\infty$的最大值。考察其中第$i$个方程,得到
\[\sum_{j=1}^na_{ij}x_j=b_i\]
考虑到$\nm{b}_\infty=1$,有限制$|b_i|\leq 1$,代入上式得
\[|a_{ii}x_i+\sum_{j=1,j\neq i}^n a_{ij}x_j|\leq 1\quad \forall 1\leq i\leq n\]
其中
\begin{align*}
LHS&\geq a_{ii}|x_i|-|\sum_{j=1,j\neq i}^n a_{ij}x_j|\\
&\geq \nm{A^{-1}}_\infty[a_{ii}-\sum_{j=1}^n|a_{ij}|]
\end{align*}
即
\[\nm{A^{-1}}_\infty\leq\frac{1}{a_{ii}-\sum\limits_{j=1,j\neq i}^n|a_{ij}|}\quad \forall 1\leq i\leq n\]
RHS取$\min$即得所需结论。
\end{proof}




\paragraph{3.}
\begin{proof}
\begin{align*}
&A^{-1}(A-B)B^{-1}\\
=&(I-A^{-1}B)B^{-1}\\
=&(B^{-1}-A^{-1})
\end{align*}
取范数得
\[\nm{A^{-1}-B^{-1}}=\nm{B^{-1}-A^{-1}}=\nm{A^{-1}(A-B)B^{-1}}\leq \nm{A^{-1}}\nm{B}^{-1}\nm{A-B}\]
得证。
\end{proof}

\paragraph{4.}
\begin{proof}
对阶数$n$作归纳。

\textbf{Case 1:}$n=1$时,$L=[1], D=[a_{11}]$即为所求。

\textbf{Case 2:}设$n-1$阶的情形已被证明,则对于$n$阶的情形:
既知$\Delta_i\neq 0$,故$a_{11}\neq 0$。故$A=\begin{bmatrix}a_{11}&\alpha^T\\ \alpha&\tilde{A}\end{bmatrix}$。

易由初等变换得$A$的如下分解:
\[A=\begin{bmatrix}1&0\\\frac{\alpha}{a_{11}}&I_{n-1}\end{bmatrix}\begin{bmatrix}a_{11}&0\\0&\tilde{A}\frac{1}{a_{11}}\alpha\alpha^T \end{bmatrix}\begin{bmatrix}1&0\\-\frac{\alpha}{a_{11}}&I_{n-1}\end{bmatrix}^T\]
%今设$\hat{L}=\begin{bmatrix}1&0\\-\frac{\alpha}{a_{11}}&I_{n-1}\end{bmatrix}$
%则$A=\hat{L}\hat{A}\hat{L}^T\quad \hat{A}=\begin{bmatrix}a_{11}&0\\0&\tilde{A}-\frac{1}{a_{11}}\alpha\alpha^T \end{bmatrix}$。

注意到$\tilde{A}-\frac{1}{a_{11}}\alpha\alpha^T$是$n-1$阶对称矩阵。记$\Delta^i$为其第$i$个主子式,则$\Delta^i=\frac{\Delta_i}{a_{11}}\neq 0$,故其有分解$\tilde{A}=\tilde{L}\tilde{D}\tilde{L}^T$。

综上,$A$有下述分解:
\[A=\begin{bmatrix}1&0\\\frac{\alpha}{a_{11}}&I_{n-1}\end{bmatrix}
\begin{bmatrix}1&0\\0&\tilde{L}\end{bmatrix}
\begin{bmatrix}a_{11}&0\\0&\tilde{D}\end{bmatrix}
\begin{bmatrix}1&0\\0&\tilde{L}\end{bmatrix}^T
\begin{bmatrix}1&0\\\frac{\alpha}{a_{11}}&I_{n-1}\end{bmatrix}^T\]
取$L=\begin{bmatrix}1&0\\\frac{\alpha}{a_{11}}&I_{n-1}\end{bmatrix}\begin{bmatrix}1&0\\0&\tilde{L}\end{bmatrix}=\begin{bmatrix}1&0\\\frac{\alpha}{a_{11}}&\tilde{L}\end{bmatrix}$,$D=\begin{bmatrix}a_{11}&0\\0&\tilde{D}\end{bmatrix}$即为所求之分解。
\end{proof}

\paragraph{5.}
记$D=\mathrm{diag}\{d_1,\cdots, d_n\}$,$L=I+\sum_{i>j}l_{ij}E_{ij}$,其中$E_{ij}$是仅在$(i,j)$位置为$1$,其余位置为$0$的矩阵。

注意到
\begin{enumerate}
%\item $a_{ji}=a_{ij}=d_j[\sum\limits_{k=1}^{i-1}l_{ik}l_{jk}+l_{ji}]\quad \forall i<j$
\item $a_{ij}=d_j[\sum\limits_{k=1}^{j-1}l_{ik}l_{jk}+l_{ij}]\quad \forall i>j$
\item $a_{ii}=d_i[\sum\limits_{k=1}^{i-1}l^2_{ik}+1]$
\end{enumerate}

首先进行如下操作:
\begin{enumerate}
\item $d_1=a_{11}$
\item $l_{i1}=\frac{a_{i1}}{d_1}\quad \forall 2\leq i\leq n$
\end{enumerate}
今假设$L,D$的前$(j-1)$列已构造完成,则对于第$j$列,进行如下构造:
\begin{enumerate}
\item $d_j=\frac{a_{jj}}{\sum\limits_{k=1}^{j-1}l_{jk}^2+1}$
\item $l_{ij}=\frac{1}{d_j}[a_{ij}-\sum\limits_{k=1}^{j-1}l_{ik}l_{jk}]\quad \forall j<i\leq n$
\end{enumerate}

\paragraph{6.}
选列主元的Gauss消去法误差小,因为该算法稳定性好。

改进平方根法误差小,因为相比平方根法没有进行非线性运算(开方),截断误差小。

一般的$Gauss$消去法速度最快,但四种方法在计算量级上没有本质区别。


\end{document}