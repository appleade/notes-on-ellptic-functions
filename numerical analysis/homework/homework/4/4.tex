\documentclass{ctexart}
\usepackage{amsmath,amssymb,amsthm,bm,ulem,graphicx, booktabs}
\usepackage[margin=1 in]{geometry}
\title{数值分析作业3}
\author{数91\and 董浚哲\and 2019011985}
\begin{document}
\maketitle
\newcommand{\R}{\mathbf{R}}
\newcommand{\dd}{\,\mathrm{d}}
\newcommand{\st}{\text{ s.t. }}
\newcommand{\pp}[2]{\frac{\partial #1}{\partial #2}}
\newcommand{\nm}[1]{\left\|#1\right\|}

\paragraph{1.}
\begin{proof}
$0<\omega<2$是SOR收敛的必要条件,课上已证,不再赘言。

考察$B_J$的特征值$\mu$。该特征值满足$\det(\mu I-B_J)=\det(D)^{-1}\det(\mu D-(L+U))=\det(D)^{-1}\det(\mu D+(A-D))=0$,即
\[\det(A-(1-\mu)D)=0\]

因$A$对称,故$\exists Q\in O_n\st QAQ^T=T=\mathrm{diag}\{\lambda_1,\cdots,\lambda_n\}$,其中$\lambda_1,\cdots,\lambda_n$是$A$的特征值,其中因A非异而非正定,$\exists 1\leq k\leq n\st \lambda_k<0$。又因$Q$正交,$D$是对角矩阵在正交阵的合同变换下不变,故
\[\det(T-(1-\mu)D)=0\]
考虑到其对角形式,上式等价于
\[\lambda_l-(1-\mu)a_{ll}=0\quad \exists 1\leq l\leq n\]

即$\lambda_l=(1-\mu)a_{ll}>0$。这构建了$B_J$与$A$之间特征值的双射,故$\forall\lambda\in \sigma(A),\lambda>0$,故$A$正定。
\end{proof}

\paragraph{2.}
\begin{proof}
以$e^0$为起始误差,则$e^{k+1}=Be^k$。即证$e^n=B^ne^0=0\quad\forall e_0\in \R^n$

设$B$的Jordan标准型为$J=\mathrm{diag}\{J_1,\cdots,J_k\}\quad k\leq n$,其中
\[J_i=
\begin{bmatrix}
0&1& & \\
 &\ddots&\ddots& \\
 & &0&1\\
 & & &0
\end{bmatrix}\]
不妨设$k=1$,则$\exists V\in\R^{n\times n}$非异$\st B=VJV^{-1}\Rightarrow B^n=VJ^nV^{-1}$。由$J$的形状知$J^n=0$,故$B_n=0\Rightarrow e_n=b^ne_0=0$
\end{proof}

\paragraph{3.}
\begin{proof}
注意到$\varphi(x)\geq 0$。这是因为$\delta\varphi(x)=A$正定,故$\varphi$是严格凸的函数,有唯一的极小值点作为最小值;又该最小值为$\varphi(x^*)=0$,故$\varphi(x)\geq 0$

记$\kappa=\kappa_2(A)$。

由定理4.5.2的证明知,
\[\nm{x^k-x^*}_A^2=2\varphi(x^k)+b^{-1}Ab\]
\[\nm{x^{k+1}-x^*}_A\leq\frac{\kappa-1}{\kappa+1}\nm{x^k-x^*}_A\]

因$A$正定,故$A^{-1}$正定,故记$C=b^TA^{-1}b>0$,则将上述两式整理得

\[\varphi(x^{k+1})\leq(\frac{\kappa-1}{\kappa+1})^2\varphi(x^{k})-\frac{2\kappa}{(\kappa+1)^2}C\leq(\frac{\kappa-1}{\kappa+1})^2\varphi(x^{k})\]

注意到$\kappa>1$,故$(\frac{\kappa-1}{\kappa+1})^2\backslash (1-\frac{1}{\kappa})=\frac{\kappa(\kappa-1)}{(\kappa+1)^2}<1$,故
\[\varphi(x^{k+1})\leq(1-\frac{1}{\kappa})\varphi(x^k)\]
得证。
\end{proof}


\paragraph{4.}
\begin{proof}
反设$\{p_1,\cdots,p_k\}$线性相关,则$\exists c_1,\cdots ,c_k$不全为0$\st \sum_{i=1}^k c_ip_i=0\Rightarrow \sum_{i=1}^k c_iAp_i=0\Rightarrow \sum_{i=1}^n c_ip_j^TAp_i=p_j^TAp_j=0\quad \forall 1\leq j\leq k \st c_j\neq 0$。因$A$对称正定,$p_j=0$,矛盾!

\end{proof}

\paragraph{5.}
\begin{tabular}{llll}
      & Jacobi迭代 & Gauss-Seidel迭代 & 共轭梯度法    \\
用时(秒) & 0.011937 & 0.008770       & 0.025307 \\
步数    & 35       & 59             & 5       
\end{tabular}








\end{document}