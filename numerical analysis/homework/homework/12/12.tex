\documentclass{ctexart}
\usepackage{amsmath,amssymb,amsthm,bm,ulem,graphicx, booktabs}
\usepackage[margin=1 in]{geometry}
\title{数值分析作业8}
\author{数91\and 董浚哲\and 2019011985}
\begin{document}
%\maketitle
\newcommand{\R}{\mathbf{R}}
\newcommand{\dd}{\,\mathrm{d}}
\newcommand{\st}{\text{ s.t. }}
\newcommand{\pp}[2]{\frac{\partial #1}{\partial #2}}
\newcommand{\nm}[1]{\left\|#1\right\|}

\paragraph{3.}
\begin{proof}
只需证明该分布具有与正态分布相同的特征函数即可。今对$Y_1$证明,$Y_2$的情形是类似的。

考察其特征函数:
\[\varphi(\xi)=E[e^{i\xi Y_1}]=\iint_{[0,1]^2}e^{i\xi\sqrt{-2\log(x_1)}\cos(2\pi x_2)}\dd x_1\dd x_2\]
作极坐标替换:令$x_1=e^{-\frac{r^2}{2}}\Leftrightarrow r=\sqrt{-2\log(x_1)},x_2=\frac{\theta}{2\pi}\Leftrightarrow \theta=2\pi x_2$,则
\[\varphi(\xi)=\int_0^{2\pi}\int_0^\infty e^{i\xi r\cos(\theta)}\frac{1}{2\pi}e^{-\frac{r^2}{2}}   r\dd r\dd \theta\]
再取直角坐标$u_1=r\cos(\theta),u_2=r\sin(\theta)$,则
\begin{align*}
\varphi(\xi)
&=\frac{1}{2\pi}\iint_{\R^2}e^{i\xi u_1-\frac{u_1^2+u_2^2}{2}}\dd u_1\dd u_2\\
&=e^{-\frac{\xi^2}{2}}{2}\frac{1}{\sqrt{2\pi}}\int_\R e^{\frac{(u_1-i\xi)^2}{2}}\dd u_1\frac{1}{\sqrt{2\pi}}\int_R e^{-\frac{u_2^2}{2}}\dd u_2\\
&=e^{-\frac{\xi^2}{2}}
\end{align*}
故其与标准正态分布具有相同的特征函数,得证。
\end{proof}


\end{document}