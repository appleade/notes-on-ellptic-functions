\documentclass{ctexart}
\usepackage{amsmath,amssymb,amsthm,bm,ulem}
\usepackage[margin=1 in]{geometry}
\title{数值分析笔记}
\author{数91 董浚哲}
\begin{document}
\maketitle
\newcommand{\R}{\mathbf{R}}
\newcommand{\dd}{\,\mathrm{d}}
\newcommand{\st}{\text{ s.t. }}
\newcommand{\pp}[2]{\frac{\partial #1}{\partial #2}}
\newcommand{\fl}{\mathrm{fl}}
\newcommand{\nm}[1]{\left\|#1 \right\|}
\newcommand{\dif}[2]{\frac{\mathrm{d}#1}{\mathrm{d}#2}}

\newtheorem{Thm}{定理}[section]
\newtheorem{Lemma}[Thm]{引理}
\newtheorem{Prop}[Thm]{命题}
\newtheorem{Cor}[Thm]{推论}
\newtheorem{Def}{定义}[section]
\newtheorem{Rmk}{注}[section]
\newtheorem{Eg}{例}[section]
\newenvironment{solution}{\begin{proof}[Solution]}{\end{proof}}

\section{非线性方程(组)的数值解法}
\subsection{单方程的不动点迭代法}
为求解方程$f(x)=0$,构造$\varphi(x)\st x^*=\varphi(x^*)\Leftrightarrow f(x^*)$。称$x^*$为$\varphi(x)$的不动点,相应迭代方法$x_{k+1}=\varphi(x_k)$称(单步定常)不动点迭代,$\varphi$称迭代函数。

\begin{Thm}
$\varphi\in C([a,b],[a,b])$,则其有不动点。又若$\exists L\in (0,1)\st |\varphi(x)-\varphi(y)|\leq L|x-y|$,则不动点唯一。
\end{Thm}

\begin{proof}
WLOG,设$\varphi(a)>a,\varphi(b)<b$。设$\psi(x)=\varphi(x)-x$,则$\psi(a)\psi(b)<0$,由介值定理即得。

设$x_1^*,x_2^*$是两个不动点,则$|x_1^*-x_2^*|=|\varphi(x_1^*)-\varphi(x_2^*)|\leq L|x_1^*-x_2^*|$,矛盾!
\end{proof}

\begin{Cor}
设$\varphi\in C^1,|\varphi'(x)|\leq L<1$,则$\varphi(x)$于$[a,b]$有唯一不动点。
\end{Cor}

\begin{Thm}
在上述条件下,$x_{k+1}=\varphi(x_k)$产生的序列收敛到不动点$x^*$,且有后验估计:
\[|x_{k+p}-x_k|\leq\frac{1}{1-L}|x_{k+1}-x_k|\]
\[|x^*-x_k|\leq \frac{L}{1-L}|x_k-x_{k-1}|\]
\[|x^*-x_k|\leq \frac{L^k}{1-L}|x_1-x_0|\]
\end{Thm}

\begin{proof}
1.施三角不等式拆成$p$项再以Lipschitz常数,然后再令$p\to \infty$。
\end{proof}

\begin{Def}
设$\varphi$在区间$I$上有不动点$x^*$。若$\exists x^*$的邻域$S\subset I\st\forall x_0\in S$,迭代法产生的序列$\{x_k\}\subset S,x_k\to x^*\quad(k\to\infty)$,则称该迭代法局部收敛。
\end{Def}

\begin{Thm}
设$x^*$是$\varphi$的不动点,$\varphi'$在$x^*$的某个邻域连续且$|\varphi'(x^*)|<1$,则该迭代法局部收敛。
\end{Thm}

若所构造的迭代方法是线性的,则$\varphi'$恰为迭代矩阵。

\begin{Def}
设$\lim\limits_{k\to\infty}x_k=x^*,e_k=x_k-x^*$
\begin{enumerate}
\item 若$\exists p\geq 1,C\neq 0\st$
\[\lim_{k\to\infty}\frac{|e_{k+1}|}{|e_k|}=C\]
其中$p=1$时,要求$C<1$,则称$\{x_k\}$为$p$阶收敛,$C$称为渐进误差常数。
\item 若$\exists p\geq 1,C>0,K\in\mathbf{N}\st\forall k>K$
\[|e_{k+1}|\leq C|e_k|^p\]
则称$\{x_k\}$至少$p$阶收敛。
\end{enumerate}
\end{Def}

\begin{Thm}
设$x^*$是$\varphi$的不动点,$p\in\mathbf{N},\varphi\in C^p$,且$\varphi'(x^*)=\cdots=\varphi^{(p-1)}(x^*)=0,\varphi^{(p)}(x^*)\neq 0$,则由迭代法产生的序列$\{x_k\}$在$x^*$的邻域$p$阶收敛,且$\lim\limits_{k\to\infty}\frac{|e_{k+1}|}{|e_k|^p}=\frac{\varphi^{(p)}(x^*)}{p!}$
\end{Thm}
\begin{proof}
施Taylor展开:$x_{k+1}-x^*=\varphi(x_k)-\varphi(x^*)=\frac{\varphi^{(p)}(\xi_k)}{p!}(x_k-x^*)^p$,按定义即得。
\end{proof}

\subsection{加速收敛的方法}
只能作用于已经收敛的方法,不能使不收敛者收敛。

\subsubsection{Aitken加速方法}
设$\{x_k\}$线性收敛于$x^*$。若可以置信:
\[\frac{x_{k+1}-x^*}{x_k-x^*}\sim\frac{x_{k+2}-x^*}{x_{k+1}-x^*}\]
则
\[x^*\sim\bar x_k=\frac{x_kx_{k+2}-x_{k+1}^2}{x_{k+2}-2x_{k+1}+x_k}\]
\begin{Thm}
设$\exists|\lambda|<1\st x_{k+1}-x^*=(\lambda+\delta_k)(x_k-x^*)$,其中$\lim\limits_{k\to\infty}\delta_k=0$,则对充分大的$k$,上述$\bar x_k$存在,且
\[\lim_{k\to\infty}\frac{\bar x_k-x^*}{x_k-x^*}=0\]
\end{Thm}
\begin{proof}
设$\Delta x_k=x_{k+1}-x_k=e_k[(\lambda-1)+\delta_k]$,则$\Delta^2 x_k=e_k[(\lambda-1)^2+\mu_k]\quad \mu_k<<1$。注意到$\bar x_k=x_k-\frac{(\Delta x_k)^2}{\Delta^2 x_k}$,故$\lim\limits_{k\to\infty}\frac{\bar x_k-x^*}{x_k-x^*}=\lim\limits_{k\to\infty}(1-\frac{[\lambda-1+\delta_k]^2}{(\lambda-1)^2+\mu_k})=0$,得证。
\end{proof}

\subsubsection{Steffensen迭代法}
直接将$\bar x_k$作为所保留的序列,则得Steffensen迭代法:
\begin{itemize}
\item $y_k=\varphi(x_k)$
\item $z_k=\varphi(y_k)$
\item $x_{k+1}=x_k-\frac{(y_k-x_k)^2}{z_k-2y_k+x_k}$
\end{itemize}
其迭代函数为:
\[\psi(x)=x-\frac{[\varphi(x)-x]^2}{\varphi(\varphi(x))-2\varphi(x)+x}=\frac{x\varphi(\varphi(x))-\varphi(x)^2}{\varphi(\varphi(x))-2\varphi(x)+x}\]

\begin{Thm}
若$x^*$是$\psi$的不动点,则$x^*$是$\varphi$的不动点。

若$x^*$是$\varphi$的不动点,设$\varphi'$存在连续,$\varphi'(x^*)\neq 1$,则$x^*$是$\psi$的不动点。
\end{Thm}

\begin{Thm}
设$x^*$是$\varphi$的不动点,$\varphi\in C^{p+1}$。对$p=1$,若$\varphi'(x^*)\neq 1$,则Steffensen迭代法二阶收敛;若$x_{k+1}=\varphi(x_k)\, p$阶收敛$(p>1)$,则Steffensen迭代法$2p-1$阶收敛。
\end{Thm}

\subsection{Newton迭代法与割线法}
\subsubsection{Newton迭代法}
为求解$f(x)=0$,作一阶近似$0=f(x^*)=f(x_k)+f'(x_k)(x^*-x_k)\Rightarrow x_{k+1}=x_k-\frac{f(x_k)}{f'(x_k)}$,遂得迭代函数:
\[\varphi(x)=x-\frac{f(x)}{f'(x)}\]

\begin{Thm}
设$f(x^*)=0,f'(x^*)\neq 0$,且$f$在$x^*$的邻域上有二阶连续导数,则Newton迭代法局部收敛到$x^*$,收敛速度为至少二阶收敛:
\[\lim_{k\to\infty}\frac{x_{k+1}-x^*}{(x_k-x^*)^2}=\frac{f''(x^*)}{2f'(x^*)}\]
\end{Thm}
即要求$x^*$是单根。

\begin{proof}
直接对LHS作Taylor展开。
\end{proof}

\subsubsection{重根情形}
设$f(x)=(x-x^*)^mg(x)\quad m>1,g(x^*)\neq 0,g\in C^2$,则$\varphi(x)=x-\frac{(x-x^*)g(x)}{mg(x)+(x-x^*)g'(x)}\Rightarrow \varphi'(x^*)=1-\frac{1}{m}<1$,故此时Newton迭代法局部线性收敛。

下面改进重根情形的Newton迭代法:

\paragraph{方法1}若已知$m$,则取$\varphi(x)=x-\frac{mf(x)}{f'(x)}$,则$\varphi(x^*)=x^*,\varphi'(x^*)=0$,至少二阶收敛。

\paragraph{方法2}设$\mu(x)=\frac{f(x)}{f'(x)}=(x-x^*)\frac{g(x)}{mg(x)+(x-x^*)g'(x)}$,则$x^*$是$\mu(x)$的单根。对$\mu(x)$用原始的Newton迭代法:
\[\varphi(x)=x-\frac{f(x)f'(x)}{[f'(x)]^2-f(x)f''(x)}\]
这种迭代法至少二阶收敛。但这种方法需要计算$f''(x)$,且$f(x),f'(x)$接近$0$,计算过程中出现小数与小数相除,丢失有效数字。

\subsubsection{割线法}
为避免计算$f'(x)$,用差商近似导数:
\[f'(x_k)\approx \frac{f(x_k)-f(x_{k-1})}{x_k-x_{k-1}}\]
则Newton法变为
\[x_{k+1}=x_k-\frac{f(x_k)(x_{k}-x_{k-1})}{f(x_k)-f(x_{k-1})}\]
割线法是一种多步法,不能写成$x_{k+1}=\varphi(x)$的形式,不能用上文的理论进行分析。
\begin{Thm}
设$f(x^*)=0,f'(x)$于$\Delta=B(x^*,\delta)$非退化,$f\in C^2(\Delta),M\delta<1$,其中$M=\frac{\max_{x\in \Delta}|f''(x)|}{2\min_{x\in \Delta}|f'(x)|}$,则当$x_0,x_1\in \Delta$时,割线法按$p=\frac{1+\sqrt{5}}{2}$阶收敛。
\end{Thm}

\begin{proof}
记$x_k,x_{k+1}$的割线为$l_k(x^*)$,并估计插值误差$l_k(x^*)-f(x^*)$,由此可以估计$x_{k+1}-x^*$。为此,设
\[p_1(x)=f(x_k)\frac{x-x_{k-1}}{x_k-x_{k-1}}+f(x_{k-1})\frac{x-x_k}{x_{k-1}-x_k}\]
并设
\[R(x)=P_1(x)-f(x)\]\[E(t)=R(t)-\frac{R(x^*)}{(x^*-x_{k-1})(x^*-x_k)}(t-x_{k-1})(t-x_k)\]则$E(x^*)=E(x_{k-1})=E(x_k)=0\Rightarrow \exists\xi\st $
\[E''(\xi)=-f''(\xi)-\frac{2R(x^*)}{(x^*-x_{k-1})(x^*-x_k)}=0\]
\[\Rightarrow R(x^*)=P_1(x^*)=-\frac{1}{2}f''(\xi)(x^*-x_{k-1})(x^*-x_k)\]
另一方面
\begin{align*}
P_1(x^*)&=P_1(x^*)-P_1(x_{k+1})\\
&=\frac{f(x_k)-f(x_{k-1})}{x_k-x_{k-1}}(x^*-x_{k+1})\\
&=f'(\eta)(x^*-x_{k+1})
\end{align*}
故
\[x^*-x_{k+1}=-\frac{1}{2}\frac{f''(\xi)}{f'(\eta)}(x^*-x_{k-1})(x^*-x_k)\]
即$e_{k+1}=|\frac{f''(\xi)}{f'(\eta)}|e_ke_{k-1}\leq M\delta^2<\delta\Rightarrow x_{k+1}\in\Delta,e_k\leq (M\delta)^k\delta\to 0\Rightarrow \lim\limits_{k\to\infty}x_k=x^*$。
\end{proof}

\subsection{非线性方程组的不动点迭代法}
求解方程可以转化为求解优化问题:$f(x^*)=0\Leftrightarrow x^*=\mathrm{argmin}|f(x)|^2$。可以用最速下降法求解,但步长难以确定。

设$F:D\subset \R^n\to\R^n$。将$F(x)=0$改写为等价的不动点形式$x=\Phi(x)$,其中$\Phi(x):D\to \R^n$连续,并构造相应不动点迭代$x^{k+1}=\Phi(x^k)$。考察不动点迭代的一般性质:
\begin{Def}
称$\Phi:D\to\R^n$为压缩映射,如果$\exists L\in (0,1)\st\nm{\Phi(y)-\Phi(x)}\leq L\nm{x-y}\quad\forall x,y\in D_0\subset D$
\end{Def}

\begin{Thm}
设$\Phi:D\to\R^n$,若存在凸区域$D_0\subset D\st \Phi\in C^1(D_0),\Phi(x)=(\varphi_1(x),\cdots, \varphi_n(x))^T$满足$\exists L\in (0,1),|\pp{\varphi_i(x)}{x_j}|\leq \frac{L}{n}$,则$\Phi$在$D_0$上对$\infty$范数是压缩的。
\end{Thm}
\begin{proof}
对各维分别施微分中值定理,由凸性可以用于放缩。记$A=(a_{ij}),a_{ij}=\pp{\varphi_i(x+\xi_i h)}{x_j},\nm{A}_\infty\leq L$,故$\nm{\Phi(y)-\Phi(x)}_\infty\leq \nm{A}_\infty\nm{y-x}_\infty\leq L\nm{y-x}_\infty$
\end{proof}

\begin{Thm}
设$\Phi:D_0\to \R^n$压缩,$\Phi(x)\in D_0\quad \forall x\in D_0$,则$\Phi$在$D_0$中由唯一的不动点$x^*$,不动点迭代收敛于$x^*$,且
\[\nm{x^*-x^k}\leq \frac{1}{1-L}\nm{x^{k+1}-x^k}\leq \frac{L^k}{1-L}\nm{x^1-x^0}\]
\end{Thm}

\begin{proof}
先用Cauchy收敛准则,再令$p\to\infty$,其余同一维情形。
\end{proof}

由此后验估计,我们得知$L$越小,收敛速度越快。

\begin{Def}
$\Phi :D\subset \R^n \to\R^n$,$x^*$为其不动点。若存在$x^*$的邻域$S\st\forall x^0\in S$,迭代序列$\{x^k\}\subset S,\lim\limits_{k\to\infty}x^k=x^*$,则称迭代法局部收敛。

若$\exists p\geq 1,C>0\st$
\[\lim_{k\to\infty}\frac{\nm{x^{k+1}-x^*}}{\nm{x^k-x^*}^p}=C\]
则称$\{x^k\} p$阶收敛。

若
\[\nm{x^{k+1}-x^*}\leq C\nm{x^k-x^*}^p\]
则称$\{x^{k}\}$至少$p$阶收敛。
\end{Def}

\begin{Thm}
设$\Phi:D\subset \R^n\to\R^n,x^*\in D$是$\Phi$的不动点,且$\Phi$于$x^*$可导且满足
\[\rho(\Phi'(x^*))=\sigma<1\]
则$\exists S=B(x^*,\delta)\subset D\st\forall x^0\in S$,迭代法产生的序列$\{x^k\}$收敛于$x^*$,即局部收敛。
\end{Thm}

\begin{proof}
$\nm{x^{k+1}-x^*}=\nm{\Phi(x^k)-\Phi(x^*)}\leq \nm{\Phi(x^k)-\Phi(x^*)-\Phi'(x^*)(x_k-x^*)}+\nm{\Phi'(x^*)(x^k-x^*)}=I_1+I_2$。取范数$\nm{\cdot}_\varepsilon\st \nm{\Phi'}_{\varepsilon}\leq \rho(\Phi')+\varepsilon<1$,则$I_2\leq \nm{|\Phi'(x^*)}_{\varepsilon}\nm{x^k-x^*}\leq (\rho(\Phi')+\varepsilon)\nm{x_k-x^*}$。由导数的定义,当$\nm{x^k-x^*}$充分小时,$I_1=o(\nm{x^k-x^*})\leq \varepsilon\nm{x^k-x^*}$,取$\varepsilon<\frac{1-\rho(\Phi')}{3}$即得。
\end{proof}

当迭代函数为线性时,结论与线性方程组的结论相容。

可以想象,当谱半径退化至$0$时,收敛速度将超过一阶。

\subsubsection{方程组的Newton法与拟Newton法}
在一维的情形犹可以使用二分法,但对于非线性方程组难以做到。但可以构造类似线性方程组迭代解法的方法化为1阶情形:在求解$x_i^{k+1}$时,解方程$f_i(x_1^k,\cdots,x_{i-1}^k,x_i^{k+1},x_{i+1}^k,\cdots,x_n^k)=0$.这样便可以用解一维方程的方法求解$x^{k+1}$。因其在线性情形与Jacobi迭代相容,故称其为非线性方程的Jacobi迭代。类似地,解方程$f_i(x_1^{k+1},\cdots,x_{i-1}^{k+1},x_i^{k+1},x_{i+1}^k,\cdots,x_n^k)=0$的方法称为非线性方程的Gauss-Seidel迭代。取$\bar{x}^k=\omega x^k+(1-\omega)x^{k+1}$得到非线性方程的SOR迭代。上述迭代法收敛速度较慢。

\paragraph{Newton法}作线性近似得到Newton迭代法:
\[x^{k+1}=x^k-(F'(x^k))^{-1}F(x^k)\]
其中要解线性方程组,要求$F'(x^k)$是非异的。

\begin{Thm}
设$D:D\subset \R^n\to\R^n,x^*\in D$满足$F(x^*)=0$,在$x^*$的开邻域$S_0\subset D$,$F$可导,$F'(x)$连续,$F'(x^*)$非异,则
\begin{enumerate}
\item $\exists B(x^*,\delta)\subset S_0\st$Newton迭代对$\forall x\in B(x^*,\delta)$有意义。
\item Newton迭代法产生的序列$\{x^k\}$局部收敛于$x^*$,且超线性收敛。
\item 若$\exists \gamma>0\st \nm{F'(x)-F'(x^*)}\leq \gamma{x-x^*}\quad \forall x^*\in S$(即一阶导数Lipschitz连续),则$\{x^k\}$至少平方收敛。
\end{enumerate}
\end{Thm}

\begin{proof}
\begin{enumerate}
\item 设$\alpha=\nm{F'(x^*)^{-1}}$。由连续性,$\exists \delta>0\st\forall x\in B(x^*,\delta)\subset S_0$,$\nm{F'(x)-F'(x^*)}<\frac{1}{2\alpha}\Rightarrow F'(x)$可逆,$\nm{F'(x)^{-1}}\leq\frac{\alpha}{1-\frac{\alpha}{2\alpha}}=2\alpha\quad\forall x\in B(x^*,\delta)$
\item 
\begin{align*}
&\nm{x^{k+1}-x^*}\\
\leq &2\alpha\nm{F(x^*)-F(x^k)-F'(x^k)(x^*-x^k)}\\
\leq &2\alpha[\nm{F(x^k)-F(x^*)-F'(x^*)(x-x^*)}+\nm{F'(x^*)-F'(x^k)}\nm{x^*-x^k}]\\
=&I_1+I_2
\end{align*}
在$B(x^*,\delta_1(\varepsilon))$内,$I_1=o(\nm{x^*-x^k})<\varepsilon\nm{x^*-x^k}$。

由连续性,$x\in B(x^*,\delta_2(\varepsilon))$时,$\nm{F'(x)-F'(x^*)}\leq\varepsilon$。

今取$\varepsilon=\frac{1}{\delta\alpha}$,$\tilde{\delta}=\min\{\delta_1(\varepsilon),\delta_2(\varepsilon)\}$。

若$x^k\in B(x^*,\tilde{\delta})$,则$\nm{x^{k+1}-x^*}\leq\frac{1}{2}\nm{x^k-x^*}$,故知其局部收敛,且\[\frac{\nm{x^{k+1}-x^*}}{\nm{x^k-x^*}}\leq 2\varepsilon\to 0\quad(k\to\infty)\]
故知其超线性收敛。
\item 

此时容易估计$I_2\leq\alpha\gamma\nm{x^k-x^*}^2$。

为估计$I_1$,考察单变量向量值函数\[g(t)=F(x^*+t(x^k-x^*))\]\[g'(t)=F'(x^*+t(x^k-x^*))(x^k-x^*)\]\[\nm{g'(t)-g'(0)}\leq \gamma t\nm{x^k-x^*}^2\] 故
\[I_1=\nm{g(1)-g(0)-g'(0)}
=\int_0^1 g'(t)-g'(0)\dd t
\leq \frac{1}{2}\gamma\nm{x^k-x^*}^2\]

综上$\{x^k\}$至少二阶收敛。
\end{enumerate}
\end{proof}

上述定理表明初值的选择将显著影响收敛性,但遗憾的是并没有一般的选取初值的方法。
\paragraph{拟Newton法}令$x^{k+1}=x^k-A_k^{-1}F(x^k)$,其中$A_k$是$F'(x^k)$的近似。其中要求$A_{k+1}-A_k=\Delta A_k,A_{k+1}(x^{k+1}-x^k)=F(x^{k+1}-F(x^k))$(对$n^2$个未知数设置$n$个约束)。选取$\Delta A$使得其Frobenius范数最小:
\begin{align*}
\Delta A_k&=\mathrm{argmin}_{A\in Q}\nm{A}_F\\
Q&=\{A\in \R^{n\times n}:(A_k+A)p^k=q^k\}\\
p^k&=x^{k+1}-x^k\\
q^k&=F(x^{k+1})-F(x^k)
\end{align*}

用Lagrange乘子法解得$A=\frac{(q^k-A_kp^k)(p^k)^T}{\nm{p^k}_2^2}$。因该解满足$\mathrm{rank}(A)=1$,故称为Broyden秩1方法:
\[x^{k+1}=x^k-A_k^{-1}F(x^k)\]
\[A_{k+1}=A_k+\frac{(q^k-A_kp^k)(p^k)^T}{\nm{p^k}_2^2}\]

\begin{Thm}[Sherman-Morrison公式]
\[(A+u^Tv)^{-1}=A^{-1}-\frac{A^{-1}uv^TA^{-1}}{1+v^TA^{-1}u}\]
\end{Thm}

故得逆Broyden秩1方法:
\[x^{k+1}=x^k-B_kF(x^k)\]
\[B_{k+1}=B_k+\frac{(p^k-B_kq^k)(p^k)^TB_k}{(p^k)^TB_kq^k}\]



\end{document}