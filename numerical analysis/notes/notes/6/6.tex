\documentclass{ctexart}
\usepackage{amsmath,amssymb,amsthm,bm,ulem}
\usepackage[margin=1 in]{geometry}
\title{数值分析笔记}
\author{数91 董浚哲}
\begin{document}
\maketitle
\newcommand{\R}{\mathbf{R}}
\newcommand{\dd}{\,\mathrm{d}}
\newcommand{\st}{\text{ s.t. }}
\newcommand{\pp}[2]{\frac{\partial #1}{\partial #2}}
\newcommand{\fl}{\mathrm{fl}}
\newcommand{\nm}[1]{\left\|#1 \right\|}
\newcommand{\dif}[2]{\frac{\mathrm{d}#1}{\mathrm{d}#2}}
\newcommand{\off}{\mathrm{off}}


\newtheorem{Thm}{定理}[section]
\newtheorem{Lemma}[Thm]{引理}
\newtheorem{Prop}[Thm]{命题}
\newtheorem{Cor}[Thm]{推论}
\newtheorem{Def}{定义}[section]
\newtheorem{Rmk}{注}[section]
\newtheorem{Eg}{例}[section]
\newenvironment{solution}{\begin{proof}[Solution]}{\end{proof}}

\section{矩阵特征值问题的数值方法}
\subsection{特征值的估计与扰动}
\begin{Thm}
$A=[a_{ij}]\in \mathbf{C}^{n\times n}$,则$A$的特征值$\lambda\in\bigcup_{i=1}^nD_i$,其中$D_i$为复平面上以$a_{ii}$为中心,$r_i=\sum\limits_{j=1,j\neq i}^n|a_{ij}|$为半径的圆盘。
\end{Thm}

\begin{Cor}
严格对角占优矩阵一定是非异的。
\end{Cor}

\begin{proof}
$D=\mathrm{diag}\{a_{11},\cdots,a_{nn}\},\lambda\in\sigma(A),x$为相应特征向量,则
\[(A-D)x=(\lambda I-D)x\]
不妨设$\lambda\neq a_{ii}$,则
\[\nm{(\lambda I-D)^{-1}(A-D)}_\infty\nm{x}_\infty\geq\nm{(\lambda I-D)^{-1}(A-D)x}_{\infty}=\nm{x}_\infty\]
\[\max_{1\leq i\leq n}\frac{r_i}{|\lambda-a_{ii}|}=\nm{(\lambda I-D)^{-1}(A-D)}_\infty\geq 1\]
$\Rightarrow \exists 1\leq i\leq n\st |\lambda-a_{ii}|\leq r_{i}$
\end{proof}

\begin{Thm}
在上述定理的$n$个圆盘中,有$m$个构成了一个连通区域$S$,且$S$与其余$n-m$个圆盘严格分离,则$S$中有且只有A的$m$个特征值。
\end{Thm}
即连通分支由几个圆盘构成,就有几个特征值。

\begin{proof}
WLOG,设前$m$个圆盘连通。

记号同上,记$A_\theta=\theta A+(1-\theta)D$。Fact:$A_\theta$的特征多项式的系数是$\theta$的多项式,故而为连续函数。

令$D_i(\theta)=B_{\theta r_i}(a_{ii})$,$S(\theta)=\bigcup\limits_{i=1}^m D_i(\theta)$,$\tilde S=\bigcup\limits_{i=m+1}^n D_i(\theta)$。则$D_i(\theta)\subset D_i\Rightarrow S(\theta)$与$\tilde S(\theta)$严格分离,由连续性易得。
\end{proof}

由此我们在不计算特征值的情况下得到了对特征值的估计。特别地,作相似变换不会改变特征值,但会改变圆盘。

\begin{Thm}
设$\mu\in\sigma(A+E)\in\mathbf{C^{n\times n}}$,且$\exists X\in \mathbf{C}^{n\times n}\st AX=XD,D=\mathrm{diag}\{\lambda_1,\cdots,\lambda_n\}$,则
\[\max_{\mu}\min_{\lambda\in\sigma(A)}|\lambda-\mu|\leq \nm{X^{-1}}_p\nm{X}_p\nm{E}_p\]
其中$p=1,2,\infty$。
\end{Thm}
故$\nm{X^{-1}}_p\nm{X}_p$称为$A$关于特征值问题的条件数。该值$\geq 1$,于对称阵($X$正交)取等。

\begin{proof}
WLOG,设$\mu\not\in\sigma(A)$,则$A+E-\mu I=X[D-\mu I+X^{-1}EX]X^{-1}$奇异,即$\exists y\in\mathbf{C^n}\st (D-\mu I)y=-(X^{-1}EX)y\Rightarrow y=-(D-\mu I)^{-1}(X^{-1}EX)y$。两边取p范数,整理得
\[\min_{\lambda\in\sigma(A)}|\lambda-\mu|=\nm{(D-\mu I)^{-1}}^{-1}_p\leq \nm{X^{-1}}_p\nm{X}_p\nm{E}_p\]
\end{proof}

设$\lambda$是$A$的单特征值,设$x,y$是其左、右特征向量,即$Ax=\lambda x,y^HA=\lambda y^H$。设$\nm{x}_2=\nm{y}_2=1$,则$\exists x(\varepsilon),\lambda(\varepsilon)\in C^1\st$
\[(A+\varepsilon F)x(\varepsilon)=\lambda(\varepsilon)x(\varepsilon)\quad \nm{F}_2=1\]
其中$\lambda(0)=\lambda,x(0)=x$。对$\varepsilon$求导得
\[Fx+A\dot x(0)=\dot\lambda(0)x+\lambda\dot x(0)\]
两边同左乘$y^H$得
\[\dot{\lambda}(0)=-\frac{y^HFx}{y^Hx}\]
\[|\dot\lambda(0)|\leq \frac{1}{y^Hx}\]
故称$\frac{1}{|y^Hx|}$为$A$关于特征值$\lambda$的条件数。特别地,该值$\geq 1$,于对称阵(左右特征向量相同)处取等。

%综上,对称矩阵的特征值对于扰动是稳定的。

\subsection{幂法和逆幂法}
设矩阵最大的特征值与其余特征值分离,则幂法求解该最大特征值及其对应特征向量,收敛速度受gap大小影响。设$|\lambda_1|>|\lambda_2|\geq\cdots\geq |\lambda_n|$,对应特征向量$u_1,\cdots,u_n$,$x=\sum\limits_{i=1}^n\alpha_i u_i\Rightarrow A^k x=\sum\limits_{i=1}^n\alpha_i\lambda_i^ku_i=\lambda+1^k(\alpha_1 u_1+\sum\limits_{i=2}^n(\frac{\lambda_i}{\lambda_1})^k\alpha_i u_i)\to \lambda_1^k\alpha_1u_1\quad(k\to\infty)$,这是幂法的基本思想。实施过程中为降低误差需每一步实行一次归一化。


设$z\in\R^n,\max(z)=z_i,|z_i|=\nm{z}_\infty$,则幂法的过程为
\begin{itemize}
\item $v^{(0)}\in\R^n$
\item for $k=1,2,\cdots$
\begin{align*}
z^{(k)}&=Av^{(k-1)}\\
m_k&=\max(z^{(k)})\\
v^{(k)}&=\frac{z^{(k)}}{m_k}
\end{align*}
\end{itemize}

\begin{Thm}
设$A$有$n$个特征向量$x_1,\cdots,x_n$,对应特征值$|\lambda_1|>|\lambda_2|\geq\cdots\geq |\lambda_n|$,且$v^{(0)}$于$x_1$投影非零,则$v^{(k)}\to \frac{x_1}{\max(x_1)},m_k\to \lambda_1$
\end{Thm}
\begin{proof}
\[v^{(k)}=\frac{A^kv^{(0)}}{m_km_{k-1}\cdots m_1}=\frac{A^kv^{(0)}}{\max(A^kv^{(0)})}\]
注意到$\frac{A^kv^{(0)}}{\lambda_1^k\alpha_1}=x_1+\sum\limits_{i=2}^n\frac{\alpha_i}{\alpha_1}(\frac{\lambda_i}{\lambda_1})^kx_i$,故
\[\lim\limits_{k\to\infty}\frac{A^kv^{(0)}}{\max(A^kv^{(0)})}=\lim\limits_{k\to\infty}\frac{A^kv^{(0)}}{\lambda_1^k\alpha_1}\frac{\lambda_1^k\alpha_1}{\max(A^kv^{(0)})}=\frac{x_1}{\max(x_1)}\]
\begin{align*}
m_k=&\max(z^{(k)})=\max(Av^{(k-1)})\\
=&\max(A\frac{A^{k-1}v^{(0)}}{\max(A^{k-1}v^{(0)})})\\
=&\frac{\max(A^{k}v^{(0)})}{\max(A^{k-1}v^{(0)})}\\
=&\lambda_1\frac{\max(\alpha_1x_1+\sum\limits_{i=2}^n\alpha_i(\frac{\lambda_i}{\lambda_n})^kx_i)}{\max(\alpha_1x_1+\sum\limits_{i=2}^n\alpha_i(\frac{\lambda_i}{\lambda_n})^{k-1}x_i)}\\
\to&\lambda_1\quad (k\to\infty)
\end{align*}
\end{proof}

\subsection{逆幂法}

\begin{itemize}
\item $v^{(0)}\in\R^n$
\item for $k=1,2,\cdots$
\begin{align*}
Az^{(k)}&=v^{(k-1)}\\
m_k&=\max(z^{(k)})\\
v^{(k)}&=\frac{z^{(k)}}{m_k}
\end{align*}
\end{itemize}
该逆幂法求解最小特征值及其特征向量。在此基础上,将幂法用于$(A-qI)^{-1}$,可得原点位移的逆幂法:
\begin{itemize}
\item $v^{(0)}\in\R^n$
\item for $k=1,2,\cdots$
\begin{align*}
(A-qI)z^{(k)}&=v^{(k-1)}\\
m_k&=\max(z^{(k)})\\
v^{(k)}&=\frac{z^{(k)}}{m_k}
\end{align*}
\end{itemize}

$q$不宜距所需特征值过近,否则矩阵刚性过强,计算误差过大。

\subsection{收缩方法}
设$|\lambda_1|>|\lambda_2|>|\lambda_3|\geq\cdots\geq|\lambda_n|$,收缩方法用于计算$\lambda_1$后计算$\lambda_2$。收缩法欲构造缩小一维的方阵使得特征值不变(只是去除$\lambda_1$)。

\paragraph{方法1} 设$\lambda_1,x_1$已知\footnote{特征向量已归一化},取Householder变换$H\st Hx_1=e_1$。由$Ax_1=\lambda_1x_1$,得$HAH^{-1}e_1=\lambda_1e_1$,即$A_2=HAH^{-1}=\begin{bmatrix}\lambda_1&b_1^T\\0&B_2\end{bmatrix}$,其中$B_2$具特征值$\lambda_2,\cdots,\lambda_n$。设$B_2y_2=\lambda_2y_2$,则$A_2$对应的特征向量为$z_2=[\frac{b_1^Ty}{\lambda_2-\lambda_1},y_2]^T$,$A$对应特征向量$x_2=H^{-1}z_2$,以此类推。

\paragraph{方法2} 将已算出的特征值归零。
\begin{Thm}
设$\lambda_1$的代数重数为1,$v^Tx_1=1$,则$B=A-\lambda_1x_1v^T$的特征值为$0,\lambda_2,\cdots,\lambda_n$,特征向量为$x_1,y_2,\cdots,y_n$,且$x_i=(\lambda_i-\lambda_1)y_i+\lambda_i(v^Ty_i)x_1\quad i=2,\cdots,n$
\end{Thm}

由此得到Wielandt收缩方法:设$x_{1i}\neq 0$,取$v=\frac{1}{\lambda x_{1i}}(a_{i1},\cdots,a_{in})^T$,则$v^Tx_1=1$,且$B=A-\lambda_1x_1v^T$的第$i$行为0。删去第i行、i列得到$B'\in\R^{(n-1)\times (n-1)}$,以幂法计算其特征值$\lambda_2$,特征向量$y'_2$,于第i个分量插入0得到$y'_2$,再由上述定理中的公式即得所需特征向量。

\subsection{QR方法}
基本的QR方法为:
\begin{itemize}
\item $A_1=A$
\item for $k=1,2,\cdots$
\begin{align*}
A_k&=Q_kR_k\\
A_{k+1}&=R_kQ_k
\end{align*}
\end{itemize}

\begin{Thm}
QR算法产生的$A_k$满足
\begin{enumerate}
\item $A_{k+1}=Q_k^TA_kQ_k$
\item $\bar Q_k^TA\bar Q_k\quad \bar Q_k=Q_1Q_2\cdots Q_k$
\item $A^k=\bar Q_k\bar R_k\quad \bar R_k=R_kR_{k-1}\cdots R_1$
\end{enumerate}
\end{Thm}
\begin{proof}
用数学归纳法直接验证。
\end{proof}

\begin{Def}
若$\{A_k\}$的对角元素收敛,严格下三角部分收敛至$0$,(严格上三角部分可以不收敛),则称$\{A_k\}$基本收敛到上三角矩阵
\end{Def}

\begin{Thm}
$A\in\R^{n\times n}$的特征值代数重数均为1,有完备的特征向量组$X=[x_1,\cdots,x_n]$,$X^{-1}=LU$,其中$L$为单位下三角矩阵,则QR算法产生的序列$A_k$基本收敛于上三角矩阵,且对角元的极限为
\[\lim_{k\to\infty}a_{ii}^{(k)}=\lambda_i\]
\end{Thm}

\subsection{上Hessenberg矩阵}
为降低QR分解的计算量,我们引入:
\begin{Def}
称$B\in\R^{n\times n}$为上Hessenberg矩阵,若$\forall i>j+1,b_{ij}=0$。称不可约的上Hessenberg矩阵,若$\forall i=2,\cdots,n, b_{i,i-1}\neq 0$。
\end{Def}
\begin{Thm}
$A\in\R^{n\times n}$,存在$Q\in O(n)\st B=Q^TAQ$为上Hessenberg矩阵。
\end{Thm}

既得到上Hessenberg矩阵$H$,对$H$施QR分解以Givens变换,则
\[U^T=J(n-1,n,\theta_{n-1})\cdots J(1,2,\theta_1),H=UR\]
易见$U$是上Hessenberg矩阵,故$H_2=RU$犹为上Hessenberg矩阵。


\begin{Thm}[隐式Q定理]
$A=\R^{n\times n},Q=[q_1,\cdots, q_n],V=[v_1,\cdots,v_n]\in O(n)$,且$Q^TAQ=H,V^TAV=G$是上Hessenberg矩阵。记$k=\mathrm{argmin}\{h_{k+1,k}=0\}$若$q_1=v_1$,则$q_i=\pm v_i\quad i=2,\cdots n, |h_{i,i-1}|=|g_{i,i-1}|$,且若$k<n$,则$g_{k+1,k}=0$
\end{Thm}

\begin{proof}
记$W=V^TQ$,只需说明$W$是单位阵。

易知$GW=WH$,则对于$i=2,\cdots,n$,比较上式的第$i-1$列:
\[Gw_{i-1}=[w_1,\cdots, w_n](h_{1,i-1},\cdots,h_{i,i-1},0,\cdots,0)\]
整理得
\[h_{i,i-1}w_i=Gw_{i-1}-\sum_{j=1}^{i-1}h_{j,i-1}w_j\]
既知$v_1=q_1\Rightarrow w_1=e_1$。利用上式可知$w_k\in\mathrm{Span}\{e_1,\cdots,e_k\}$即$W$是上三角矩阵。又以其为正交矩阵,$w_i=\pm e_i\Rightarrow q_i=vw_i=vv^Tq_i=\pm v_i,|h_{i,i-1}|=|w_i^TGw_{i-1}|=|q_iVGV^Tq_{i-1}|=|q_i^TAq_{i-1}|=|v_i^TAv_{i-1}|=|g_{i,i-1}|$
%to be continued
\end{proof}

\begin{Cor}
若$Q^TAQ=H,V^TAV=G$均为不可约上Hessenberg矩阵,且$Q,V$的第一列相同,则$G=D^{-1}HD$,其中$D$是对角线元素为$\pm 1$的对角矩阵。
\end{Cor}


\subsection{带原点位移的QR方法}
\begin{itemize}
\item $H_1=H$
\item for $k=1,2,\cdots$
\[H_k-\mu_kI=U_kR_k\]
\[H_{k+1}=R_kU_k+\mu_kI\]
\end{itemize}
$H_{k+1}=U_k^TH_kU_k$,故迭代过程犹产生相似的矩阵。

$\mu_k$的取法为
\begin{enumerate}
\item $\mu_k=h^{(k)}_{nn}$,但在复特征值的情形下这种位移不会加速收敛。
\item $B=\begin{bmatrix}h^{(k)}_{n-1,n-1}&h^{(k)}_{n-1,n}\\h^{(k)}_{n,n-1}&h^{(k)}_{n,n}\end{bmatrix}$的两个实特征值中接近$h_{n,n}^{(k)}$的一个。
\end{enumerate}

为避免上述选取位移过程中可能出现的复数,采用:
\subsection{双重步QR方法}
设$\mu_1,\mu_2$是$\begin{bmatrix}h^{(k)}_{n-1,n-1}&h^{(k)}_{n-1,n}\\h^{(k)}_{n,n-1}&h^{(k)}_{n,n}\end{bmatrix}$的共轭复特征值,分别以其为位移作QR迭代得$H_1,H_2$。设$\mathrm{Tr}(B)=s,\det(B)=t$。设
\[M=(H-\mu_1 I)(H-\mu_2 I)=H^2-sH+tI\],则
\begin{align*}
M&=(U_1R_1+(\mu_2-\mu_1)I)(H-\mu_1I)\\
=&U_1(R_1U_1+(\mu_1-\mu_2)I)U^H(H-\mu_1 I)\\
=&U_1(H-\mu_2 I)R_1\\
=&(U_1U_2)(R_2R_1)=:ZR
\end{align*}
且
\[H_2=U_2^HH_1U_2=(U_1U_2)^HH(U_1U_2)=Z^THZ\]
故得
\begin{itemize}
\item $M=H^2-sH+tH$
\item $M=ZR$
\item $H_2=Z^THZ$
\end{itemize}

\begin{Thm}
$H$是不可约的上Hessenberg矩阵,$\mu_1,\mu_2$不是$H$的特征值,则$H_2$犹为不可约上Hessenberg矩阵。
\end{Thm}

为避免计算$H^2$的巨大计算量,希望寻求替代方法计算$M$。为此应用隐式Q定理:希望找到$Z_1\in O(n)\st Z_1(:,1)=Z(:,1),Z_1^HZ_1$是不可约上Hessenberg矩阵,则由隐式Q定理,$H_2=Z_1^THZ_1$即为所求。

故得算法:
\begin{itemize}
\item 计算$Me_1$(因QR分解之结构,Z的第一列与M的第一列平行)
\item 计算Householder矩阵$P_0\st P_0(Me_1)=\pm\nm{Me_1}_2 e_1$。如此得到的$P_0$的第一列与$Z$的第一列平行。
\item 计算Householder矩阵$P_1,\cdots,P_{n-2}$将$P_0^THP_0$变为上Hessenberg矩阵。
\item 取$Z=P_0P_1\cdots P_{n-2}$即为所求。这是因为$P_1,\cdots,P_{n-2}$总保持第一列不变。
\end{itemize}

注意到$H$是上Hessenberg矩阵(第一列仅有前两个元素非零$\Rightarrow H^2$第一列仅前三个元素非零),并注意到$M$的结构:

1.容易显式计算$Me_1$

2.$P_0$仅是三阶的Householder变换,$P_0^THP_0$仅在$(3,1),(4,1),(4,2)$三个位置引入了非零元,并在之后的Householder变换中保持结构,每一步只需做3维的Householder变换。 

\begin{Rmk}
故QR迭代是不做QR分解的。
\end{Rmk}


\subsection{实对称矩阵计算特征值的方法}
若$A$是实对称矩阵,则

1.其对应Hessenberg矩阵是三对角矩阵(且构造过程只需一半的计算量)。

2.特征值都是实数,单步位移便是足够的。

3.可以利用隐式Q定理寻求$\bar{U}_k^TH_k\bar U_k=\tilde{H}_{k+1}$,其构造过程同上:只将$Me_1$替换为$H_k-\mu_k I$的第一列,该过程全程只做2维的Householder变换。


\subsubsection{SVD的计算}
在计算奇异值的过程中需要用到$A^TA$,故可以使用上述过程。%此时注意到:事实上可以采用如下方法
%\[\begin{bmatrix}0&A^*\\ A&0\end{bmatrix}\begin{bmatrix}V&V\\U&-U\end{bmatrix}=\begin{bmatrix}V&V\\U&-U\end{bmatrix}\begin{bmatrix}\Sigma^*&0\\0&-\Sigma\end{bmatrix}\]

\textbf{Step 1:}寻$P,Q\st B=PAQ^T$是上二对角矩阵($\Rightarrow B^TB=AA^TAQ^T$是三对角矩阵)

\textbf{Step 2:}对第$k$步迭代,选取合适的位移$\mu_k$,计算$P_0\st P_0((B_k^TB_k-\mu_k I)e_1)=\nm{B^TBe_1}e_1$。

若采取显式的计算,则下面需要选取$P_1,\cdots, P_{n-2}\st Q=P_0\cdots P_{n-2},Q_k^TB_k^TB_kQ_k$是三对角矩阵。但这样需要显式构造$B^TB$,遂采用隐式方法:

\textbf{Step 3:}计算$BP_0$。注意到$P_0$是二维的Householder变换,只在$(2,1)$处引入了非零元。用下面的过程将$B_kP_0$变回上二对角矩阵:
\begin{itemize}
\item $U_1\in O(m)\st U_1B_kP_0$用Householder变换将$(2,1)$处元素变为0,同时在(1,3)处引入非零元。
\item $V_2\in O(n)\st U_1B_kP_0V_2$用Householder变换将$(3,1)$处元素变为0而引入$(2,3)$处非零元
\item ……
\item 重复此过程至$U_{n-1}\cdots U_1 B_kP_0V_2\cdots V_{n-1}$为上二对角矩阵。
\end{itemize}
此时记
\[U=U_{n-1}\cdots U_1,V=P_0V_2\cdots V_{n-1}\]则$UB_kV$是上二对角矩阵,且$Ve_1=Qe_1$(这是由$V_k,2\leq k\leq n-1$的结构得出的),且$V^TB_k^TB_k V$为三对角矩阵。由隐式Q定理,$Q_k=V\mod \pm 1$,遂得
\[Q_k=V\qquad B_{k+1}=UB_k V\]
是为迭代过程。

\subsubsection{Rayleigh商迭代法}
\begin{Def}
$A\in \mathbf{C}^{n\times n},x\in \mathbf{C}^n,x\neq 0$,则
\[R(x)=\frac{(Ax,x)}{x,x}\]
称为$A,x$的Rayleigh商。
\end{Def}
在原点位移的逆幂法中的每步选取$\mu_k=R(x^{(k)})$,则得到Rayleigh商迭代法:
\begin{itemize}
\item $v^{(0)}\in\R^n,\nm{v^{(0)}}=1$
\item for $k=0,1,\cdots$
\begin{itemize}
\item $\mu_k=R(v^{(k)})$
\item $(A-\mu_kI)y^{(k+1)}=v^{(k)}$
\item $v^{(k+1)}=\frac{y^{(k+1)}}{\nm{y^{(k+1)}}}$
\end{itemize}
\end{itemize}

\subsubsection{Jacobi方法}
对于对称矩阵$A$,寻找$P\in O(n)\st P^TAP=D$是对角阵。记$off(A)=\sum_{i,j=1,i\neq j}^n|a_{ij}|^2$,则希望$off(A)$在迭代中减小。为此考察Givens变换$J(p,q,\theta)$,$B=J^{-1}AJ$。记$\sin(\theta)=s,\cos(\theta)=c$,则$B$的元素为(非第$i,j$行、列的保持不变,略):
\begin{align*}
b_{pp}&=a_{pp}c^2+a_{qq}s^2+2a_{pq}sc\\
b_{qq}&=a_{pp}s^2+a_{qq}c^2-2a_{pq}sc\\
b_{pq}=b_{qp}&=(a_{qq}-a{pp})sc+a_{pq}(c^2-s^2)\\
b_{ip}=b_{pi}&=a_{ip}c+a_{iq}s\\
b_{iq}=b_{qi}&=-a_{ip}s+a_{iq}c
\end{align*}
故
\[b_{pp}^2+b_{qq}^2+2b_{pq}^2=a_{pp}^2+a_{qq}^2+2a_{pq}^2\]
故
\[\off(B)=\nm{B}_F^2-\sum_{i=1}^n b_{ii}^2=\nm{A}_F^2-\sum_{i=1,i\neq p,q}|a_{ii}|^2-(b_{pp}^2+b_{qq}^2)=\off(A)-2a_{pq}^2+2b_{pq}^2\]
为此希望
\begin{itemize}
\item 使得$b_{pq}=0\Rightarrow \theta\st \cot(2\theta)=\frac{a_{pp}-a_{qq}}{2a_{pq}}=\tau$。解$t=\tan(\theta)$而取绝对值较小者$t=\frac{\mathrm{sign}(\tau)}{|\tau|+\sqrt{1+\tau^2}}\Rightarrow c=\frac{1}{\sqrt{1+t^2}},s=tc,|t|<1$

\item $|a_{pq}|=\max\limits_{1\leq i<j\leq n}|a_{ij}|$使得收敛最快:$\off(A_{k+1})=\off(A_k)-2a_{pq}^2\leq (1-\frac{2}{n(n-1)})\off(A_k)\Rightarrow \lim\limits_{k\to\infty} A_k=\mathrm{diag}\{\lambda_1,\cdots,\lambda_n\}$
\end{itemize}

\begin{Lemma}
记$A,E,A+E$的特征值为$\lambda_i,\nu_i,\mu_i$,则
\[\lambda_i+\nu_n\leq\mu_i\leq \lambda_i+\nu_1\]
\end{Lemma}

\begin{Thm}
$\lim\limits_{k\to\infty} A_k=\mathrm{diag}\{\lambda_1,\cdots,\lambda_n\}$
\end{Thm}

\begin{proof}
记$E(A_k)=\sqrt{\off(A_k)}$,取$\delta=\min\{\lambda-\mu:\lambda,\mu\in\sigma(A), \lambda>\mu\}$,则$\exists N_0\st \forall k>N_0,E(A_k)<\varepsilon<\frac{\delta}{4}$。又$|\lambda_i-a^{(N_0)}_{ii}|<\varepsilon$,施数学归纳法可证$|\lambda_i-a^{(k)}_{ii}|<\varepsilon$
\end{proof}

这一方法的缺点在于收敛速度慢,优点在于可以并行运算。



\end{document}