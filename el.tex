\documentclass{article}
\usepackage{amsmath,amsthm,amssymb,ulem,bm,hyperref,color}
\usepackage[margin = 1 in]{geometry}

\title{Notes on Elliptic Functions}
\author{Junzhe Dong\and M91\and 2019011985}
\date{\today}

\newtheorem{Thm}{Theorem}[section]
\newtheorem{Lem}[Thm]{Lemma}
\newtheorem{Prop}[Thm]{Proposition}
\newtheorem{Cor}[Thm]{Corollary}
\newtheorem{Def}{Definition}[section]
\newtheorem{Rmk}{Remark}[section]
\newtheorem{Eg}{Example}[section]

\begin{document}

\maketitle

Special functions: elliptic function, theta function, polylogarithm function, hypergeo function.

You'll frequently see their modular, Jacobi and automorphic forms.

On one hand , we study Ramanujan's approach (5-7); On the other, we study Weierstrass's approach.

In Ramanujan's approach, $q$-series are important.

Referrence: Venkatachliengar Covper \textit{Development of elliptic functions according to Ramanujan}.

\section{Some basic identities}
$q$ stands for a complex variable s.t.  $|q|<1$

$Q= 1 + 240 \sum\limits_{j=1}^{\infty} \frac{j^3 q^j}{1-q^j}$

$R= 1 - 504 \sum\limits_{j=1}^{\infty} \frac{j^5q^j}{1- q^j}$

$P= 1 - 24 \sum\limits_{j=1}^{\infty}\frac{jq^j}{1-q^j}$

\begin{Prop}[Ramanujan's DE]
\begin{align*}
  q \frac{dP}{dq} &= \frac{P^2-Q}{12}\\
  q \frac{dQ}{dq} &=\frac{PQ-R}{3}\\
  q \frac{dR}{dq} &= \frac{PR-Q^2}{2}
\end{align*}

\end{Prop}

A classical identity:

\begin{equation}
\label{eq:0.id}
  Q^3-R^2=1728q\prod_{j=1}^{\infty}(1-q^j)^{24}
\end{equation}

Generalization:
\[\rho_1(z)= \frac{1}{2}+ \sum_n'\frac{z^n}{1-q^n}\] Here $\sum'$ means summing while omitting 0.

\[\rho_2(z)= -\frac{1}{12} \sum_n^{} \frac{q^nz^n}{1-q^n}\]

In case $|q|<|z|<1, \rho_1(z)$ converges. In case $|q|<|z|<|q|^{-1}, \rho_2(z)$ converges.

Given $\alpha, \beta,\gamma$ st $\alpha\beta\gamma=1$, $|q|<|\alpha|<1, |q|<|\beta|<1, |q|<|\alpha\beta|<1$, then\footnote{HW1.1}

\begin{equation*}
\rho_1(\alpha)\rho_1(\beta)-\rho_1(\alpha\beta)(\rho_1(\alpha)+ \rho_1(\beta))= \rho_2(\alpha)+ \rho_2(\beta)+ \rho_2(\gamma)
\end{equation*}

\textcolor{blue}{
 \begin{proof}
   Denote the Laurent expansion of LHS:
   \[LHS= \sum\limits_{m,n=-\infty}^{\infty} c_{m,n}\alpha^m\beta^n\]
   Then we calculate the coefficients $c_{m,n}$ by distinguishing 4 separate cases:    
\begin{enumerate}
\item $mn(m-n)\neq 0$, then
\begin{align*}
 c_{m,n} =& \frac{1}{(1-q^m)(1-q^n)} - \frac{1}{(1-q^n)(1-q^{m-n})} - \frac{1}{(1-q^m)(1-q^{n-m})} = 0\\
 =& (1-q^m)^{-1}(1-q^n)^{-1}(q^m-q^n)^{-1} \left[ (q^m-q^n) + (1-q^m)q^n -(1-q^n)q^m \right]\\
 =&0
\end{align*}
\item $m\neq 0, n=0$, then
 \begin{equation*}
c_{m,0} = \frac{1}{2(1-q^m)} - \frac{1}{2(1-q^m)} - \frac{1}{(1-q^m)(1-q^{-m})} = \frac{q^m}{(1-q^m)^2}
\end{equation*}
Similarly, $c_{0,n} = \frac{q^n}{(1-q^n)^2}$
\item $m=n\neq 0$, then
 \begin{equation*}
c_{m,m} = \frac{1}{(1-q^m)^2} - \frac{1}{2(1-q^m)} - \frac{1}{2(1-q^m)} = \frac{q^m}{(1-q^m)^2}
 \end{equation*}
\item $m=n=0$, then
 \begin{equation*}
c_{0,0} = \frac{1}{4}- \frac{1}{2} \left( \frac{1}{2} + \frac{1}{2} \right) = -\frac{1}{4}
 \end{equation*}
\end{enumerate}
Observe that $\frac{q^{-m}}{(1-q^{-m})^2} = \frac{q^m}{(1-q^m)^2}$ and that $\gamma = \alpha^{-1}\beta^{-1}$, we get the expansion of RHS:
\begin{equation*}
RHS= -\frac{1}{4} + \sum\limits_n^{'} \frac{q^n}{(1-q^n)^2} \left[ \alpha^n + \beta^n + (\alpha\beta)^n \right]
\end{equation*}
Compare the coefficients and we thus completes the proof.
 \end{proof}
}


\begin{Prop}

\begin{itemize}
\item
Express $\rho_1(z)$ in a symmetric form:
 \begin{align*}
 \rho_1(z) &= \frac{1}{2} + \sum\limits_{m=1}^{\infty}\frac{z^m}{1-q^m} + \sum\limits_{m=1}^{\infty} \frac{z^{-m}}{1-q^{-m}}\\
           &= \frac{1}{2} + \sum\limits_{m=1}^{\infty}\frac{z^m}{1-q^m} + \sum\limits_{m=1}^{\infty}z^{-m}(1- \frac{1}{1-q^m}) \\
           &= \frac{1}{2} + \sum\limits_{m=1}^{\infty}z^{-m} + \sum\limits_{m=1}^{\infty} \frac{1}{1-q^m}(z^m-z^{-m})\\
           &= \frac{1}{2} + \sum\limits_{m=1}^{\infty} (z^m-z^{-m}) + \sum\limits_{m=1}^{\infty}z^{-m} + \sum\limits_{m=1}^{\infty} \frac{q^m}{1-q^m}(z^m-z^{-m})\\
           &= \frac{1+z}{2(1-z)} + \sum\limits_{m=1}^{\infty}\sum\limits_{n=1}^{\infty}(z^mq^{mn}-z^{-m}q^{mn})\\
 &= \frac{1+z}{2(1-z)} + \sum\limits_{n=1}^{\infty} \left( \frac{zq^n}{1-zq^n} - \frac{z^{-1}q^n}{1-z^{-1}q^{n}} \right)
\end{align*}  
\item $\rho_1(z)$ has a single pole at $z=q^r, r= 0, \pm 1,\dots$
\item symmetry : 
$\rho_1(z^{-1})= \frac{1+z^{-1}}{2(1-z^{-1})} + \sum\limits_{n=1}^{\infty}\left( \frac{z^{-1}q^n}{1-z^{-1}q^n} - \frac{zq^n}{1-zq^n} \right) = - \rho_1(z)$
\item quasi-periodity :

$\rho_1(z)- \rho_1(qz)= -1$.
\begin{proof}

\begin{align*}
 \rho_1(qz)=& \frac{1+qz}{2(1-qz)} + \sum\limits_{n=1}^{\infty} \left( \frac{(qz)q^n}{1-(qz)q^{n}} - \frac{(qz)^{-1}q^n}{1-(qz)^{-1}q^n} \right)\\
 =&\frac{1+qz}{2(1-qz)} + \sum\limits_{n=1}^{\infty} \left( \frac{zq^{n+1}}{1-zq^{n+1}}- \frac{z^{-1}q^{n-1}}{1-z^{-1}q^{n-1}} \right)\\
 =& \frac{1}{2} + \frac{1}{1-z} + \sum\limits_{n=1}^{\infty} \left[\frac{zq^n}{1-zq^n} - \frac{z^{-1}q^n}{1-z^{-1}q^n} \right]\\
 =&\rho_1(z)+1
\end{align*}

\end{proof}
\end{itemize}
\end{Prop}
\begin{Prop}\footnote{HW1.2}
\begin{itemize}
\item $\rho_2(z)=\rho_2(z^{-1})$
\item $\rho_2(z)-\rho_2(qz)= \frac{1}{2}  + \rho_1(z)$
\end{itemize}
\end{Prop}

\textcolor{blue}{
\begin{proof}
\begin{align*}
 \rho_2(z) = & -\frac{1}{12} + \sum\limits_{m=1}^{\infty} \frac{q^m}{(1-q^m)^2} (z^m + z^{-m})\\
 = & -\frac{1}{12} + \sum\limits_{m=1}^{\infty}\sum\limits_{n=1}^{\infty} nq^{mn} (z^m + z^{-m})\\
 = & -\frac{1}{12} + \sum\limits_{n=1}^{\infty} \left( \frac{nzq^n}{1-zq^n} + \frac{nz^{-1}q^n}{1-z^{-1}q^n} \right)
\end{align*}
So 
\begin{enumerate}
\item symmetry:
\begin{equation*}
\rho_2(z^{-1}) = -\frac{1}{12} + \sum\limits_{n=1}^{\infty} \left( \frac{nz^{-1}q^n}{1-z^{-1}q^n} + \frac{nzq^n}{1-zq^n} \right)
\end{equation*}
\item quasi-periodic:  
\begin{align*}
 \rho_2(qz)= & \sum\limits_{n=1}^{\infty} \left( \frac{nzq^{n+1}}{1-zq^{n+1}} + \frac{nz^{-1}q^{n-1}}{1-z^{-1}q^{n-1}} \right)\\
 = & \sum\limits_{n=1}^{\infty}\left( \frac{nzq^n}{1-zq^n} - \frac{zq^n}{1-zq^n}  \right) + \sum\limits_{n=1}^{\infty} \left( \frac{nz^{-1}q^n}{1-z^{-1}q^n} + \frac{z^{-1}q^n}{1-z^{-1}q^n} \right) + \frac{1}{1-z}\\
 =& \rho_2(z) - ( \frac{1}{2} + \rho_1(z))
\end{align*}
\end{enumerate}
\end{proof}
}


Notes:
\begin{equation}
 \label{eq:1.note}
\rho_1(\alpha)\rho_1(\beta)+\rho_1(\beta)\rho_1(\gamma)+ \rho_1(\gamma)\rho_1(\alpha)= \rho_2(\alpha)+ \rho_2(\beta)+\rho_2(\gamma)
\end{equation}
where $\alpha,\beta,\gamma$ are complex numbers where $\alpha\beta\gamma=1$.


The limit case where $\beta\to 1$:

 Recall that $\rho_1(z)= \frac{1}{2} + \frac{z}{1-z} + \sum\limits_{m=1}^{\infty} \frac{q^m}{1-q^m}(z^m-z^{-m})$. $\rho_1$ has a single pole at $z=1$ with residue $1$: $\lim_{\beta\to 1}(1-\beta)\rho_1(\beta)=1$. So
\begin{align*}
 &\lim_{\beta\to 1}\rho_1(\beta)(\rho_1(\alpha)-\rho_1(\alpha\beta))\\
 =& \lim\limits_{\beta\to 1}  (1-\beta)\rho_1(\beta) \frac{\rho_1(\alpha)- \rho_1(\alpha\beta)}{1-\beta}\\
 =& \rho_1(\alpha x)'|_{x=1}\\
 =& \alpha\rho_1'(\alpha)
\end{align*}

Recall
\begin{align*}
  &\rho_1(\alpha)\rho_1(\beta) - \rho_1(\alpha\beta)(\rho_1(\alpha)+ \rho_1(\beta)) \\
  =& \rho_2(\alpha)+\rho_2(\beta)+\rho_2(\gamma)\\
  =& \rho_1(\beta)(\rho_1(\alpha)-\rho_1(\alpha\beta)) - \rho_1(\alpha)\rho_1(\alpha\beta)\\
  \to& \alpha\rho'_1(\alpha)- \rho_1^2(\alpha)\\
  =& 2\rho_2(\alpha)+\rho_2(1)
\end{align*}

\begin{Def}
 \[\phi_1(\theta)= \frac{1}{2i}\rho_1(e^{i\theta})\]
 \[\phi_2(\theta) = \frac{1}{2}\rho_2(e^{i\theta})\]
\end{Def}
Calculating their Fourier series and we have:
\[\phi_1(\theta)= \frac{1}{4}\cot(\frac{\theta}{2}) + \sum\limits_{m=1}^{\infty} \frac{q^m}{1-q^m} \sin(m\theta)\]
\[\phi_2(\theta) = -\frac{1}{24} + \sum\limits_{m=1}^{\infty} \frac{q^m}{(1-q^m)^2}\cos(m\theta)\]

\begin{Prop}
 Define $\tau$ s.t. $q=\exp(2\pi i \tau)$, then

\begin{itemize}
\item $\phi_1(-\theta)= - \phi_1(\theta)$
\item $\phi_1(\theta+2\pi)= \phi_1(\theta)$ 
\item $\phi_1(\theta+2\pi\tau)=\phi_1(\theta)-\frac{i}{2}$ 
\item $\phi_2(-\theta)=\phi_2(\theta)$
\item $\phi_2(\theta+2\pi)=\phi_2(\theta)$
\item $\phi_2(\theta+2\pi\tau)=\phi_2(\theta)-i\phi_1(\theta)-\frac{1}{4}$.
\end{itemize}
\end{Prop}

Thus ( assuming $a+b+c=0$ ) by dividing $\frac{1}{2i}* \frac{1}{2i}$ on both sides of \ref{eq:1.note}
\begin{equation}
 \label{eq1.1}
 \phi_1(a)\phi_{1}(b)+\phi_1(b)\phi_1(c)+\phi_1(c)\phi_1(a)=-\frac{1}{2} \left( \phi_2(a)+ \phi_2(b) + \phi_2(c) \right)
\end{equation}

\begin{equation}
 \label{eq1.2}
\phi_1^2(a) + \frac{1}{2}\phi'_1(a)= \phi_2(a) + \frac{1}{2}\phi_2(0)
\end{equation}

Set $a=a,b,c$ respectively in \ref{eq1.2} and add them up:

\begin{equation}
 \label{eq1.3}
\phi_1^2(a)+\phi_1^2(b)+\phi_1^2(c) + \frac{1}{2} \left( \phi'_1(a)+ \phi'_1(b) + \phi'_1(c) \right) = \phi_2(a) + \phi_2(b) + \phi_2(c) + \frac{3}{2}\phi_2(0)
\end{equation}

Add $2*$\ref{eq1.1} to \ref{eq1.3}:

\begin{equation}
 \label{eq1.4}
\left( \phi_1(a)+\phi_1(b)+\phi_1(c) \right)^2 + \frac{1}{2} \left( \phi'_1(a) + \phi'_1(b) + \phi'_1(c) \right) = \frac{3}{2}\phi_2(0)
\end{equation}


Apply $\frac{\partial }{\partial a} - \frac{\partial }{\partial b}$ to \ref{eq1.4} and we have

\begin{equation}
 \label{eq1.5}
2 \left( \phi_1(a)+\phi_1(b)+\phi_1(c) \right)(\phi'_1(a)- \phi'_1(b)) + \frac{1}{2} \left( \phi''_1(a)- \phi''_1(b) \right) = 0 
\end{equation}

\begin{equation}
 \label{eq1.6}
\phi_1(a)+ \phi_1(b) + \phi_1(c) = - \frac{1}{4} \frac{\phi''_1(a)- \phi''_1(b)}{\phi_1'(a) - \phi'_1(b)}
\end{equation}

\begin{equation}
 \label{eq1.7}
\frac{1}{16} \left( \frac{\phi''_1(a)- \phi''_1(b)}{\phi'_1(a)-\phi'_1(b)} \right)^2 = \frac{3}{2}\phi_2(0) - \frac{1}{2} \left( \phi'_1(a) + \phi'_1(b)+ \phi'_1(c) \right)
\end{equation}

\section{The Weierstrass elliptic function and Ramanujan's DE}

\begin{Def}[Weierstrass $\wp$ function]
 \[\wp(a) = 2 (\phi_2(0) - \phi'_1(a))\]
\end{Def}


So $ \frac{1}{16} \left( \frac{\wp'(a)-\wp'(b)}{\wp(b)-\wp(a)} \right)^2 = \frac{1}{4} \left( \wp(a)+\wp(b)+\wp(c) \right)\Leftrightarrow $
\begin{equation*}
\wp(a+b) = \frac{1}{4} \left( \frac{\wp'(a)-\wp'(b)}{\wp(a)-\wp(b)} \right)^2 - \wp(a)-\wp(b)
\end{equation*}
This is called the \textbf{addition theorem} for the Weierstrass elliptic function.

Recall that \[\phi_1(\theta) = \frac{1}{4} \cot(\frac{\theta}{2}) + \sum\limits_{n=1}^{\infty} \frac{q^n}{1-q^n} \sin(n\theta)\]
Meanwhile, \[\frac{1}{2} \cot(\frac{\theta}{2}) = \sum\limits_{n=0}^{\infty} \frac{(-1)^nB_{2n}}{(2n)!}\theta^{2n-1}\]
where $B_n$ are the Bernoulli numbers satisfying
\[\frac{\theta}{e^{\theta}-1} = \sum\limits_{n=0}^{\infty} \frac{B_n}{n!}\theta^{n}\]

The first Bernoulli numbers are (note that Bernoulli numbers with odd index does not exist apart from $B_1$)

\begin{table}
  \centering
  \caption{The first 10 Bernoulli numbers}
\begin{tabular}{ll|ll}
$B_0$ & 0               & $B_1$ & $-\frac{1}{2}$ \\
$B_2$ & $\frac{1}{6}$   & $B_3$ & 0              \\
$B_4$ & $-\frac{1}{30}$  & $B_5$ & 0              \\
$B_6$ & $\frac{1}{42}$  & $B_7$ & 0              \\
$B_8$ & $-\frac{1}{30}$ & $B_9$ & 0             
\end{tabular}
\end{table}

So
\begin{align*}
 \phi_1(\theta) =& \frac{1}{2\theta} + \sum\limits_{n=1}^{\infty} \frac{(-1)^{n-1}}{(2n-1)!} \left\{ -\frac{B_{2n}}{4n} + \sum\limits_{j=1}^{\infty} \frac{j^{2n-1}q^j}{1-q^j} \right\}\theta^{2n-1} \\
 = & \frac{1}{2\theta} -\frac{P}{24}\theta - \frac{Q}{240} \left( \frac{\theta^3}{3!} \right) - \frac{R}{504} \left( \frac{\theta^5}{5!} \right) - \cdots
\end{align*}

Similarly, we have expansions\footnote{HW1.3: calculate $\phi_2(0)$} of 
$\phi_2(\theta)=-\frac{1}{24} + \sum\limits_{j=1}^{\infty} \frac{q^j}{(1-q^j)^2}\cos(j\theta)$

\textcolor{blue}{
\begin{align*}
 \phi_2(0)= & -\frac{1}{24} + \sum\limits_{j=1}^{\infty} \frac{q^j}{(1-q^j)^2} \\
 = & -\frac{1}{24} + \sum\limits_{j=1}^{\infty} \sum\limits_{k=1}^{\infty} kq^{jk}\\
 = & -\frac{1}{24} + \sum\limits_{k=1}^{\infty} \frac{kq^k}{1-q^k} = -\frac{P}{24}
\end{align*}
}

\begin{Def}
\[\phi_{m,n}= \sum\limits_{j=1}^{\infty} \sum\limits_{k=1}^{\infty} j^m k^n q^{jk} = \phi_{n,m}\]  
\end{Def}

With this, we can express the expansion of $\phi_2(\theta)$ as
\begin{equation}
\phi_2(\theta)= -\frac{P}{24} + \sum\limits_{n=1}^{\infty} \frac{(-1)^n}{(2n)!}\phi_{2n,1}\theta^{2n}
\end{equation}

\begin{align*}
 \wp(\theta) =& 2(\phi_2(0) - \phi'_1(\theta))\\
 =& \frac{1}{\theta^2} - 2 \sum\limits_{n=1}^{\infty} \left\{ -\frac{B_{2n+2}}{4(n+1)} + \sum\limits_{j=1}^{\infty} \frac{j^{2n+1}q^j}{1-q^{j}} \right\}\theta^{2n} \\
 =& \frac{1}{\theta^2} + \frac{Q}{240}\theta^{2} + \frac{R}{12\times 504}\theta^4 + \cdots
\end{align*}

Now that we have \[(\phi_1(a)+ \phi_1(b))\phi_1(a+b)-\phi_1(a)\phi_1(b) = \frac{1}{2} \left( \phi_2(a)+\phi_2(b)+\phi_2(a+b) \right)\]
Expand both sides in powers of b, compare the coefficients of $b^2$, then we have (multiply both sides by $4$)

\begin{equation}
\label{eq:3}
2\phi_1(a)\phi''_1(a) - \frac{P}{6}\phi'_1(a) + \frac{1}{3}\phi''_1(a) = \phi''_2(a) - \phi_{1,2}
\end{equation}

Differentiate the \ref{eq1.2} and we get
\begin{equation}
2\phi_1(a)\phi_1'(a) + \frac{1}{2} \phi''_1(a) = \phi_2'(a)
\end{equation}
\begin{equation}
2(\phi'(a))^2 + 2\phi_1(a) \phi_1''(a) + \frac{1}{2}\phi_1'''(a) = \phi''_2(a)
\end{equation}
Eliminate $\phi_2''(a)$ with \ref{eq:3} and multiply by $2$:
\begin{equation}
4(\phi_1'(a))^2 + \frac{P}{3} \phi'_1(a) + \frac{1}{3} \phi'''_1(a) = 2 \phi_{1,2}
\end{equation}
Make it a square:
\begin{equation}
\left( 2\phi_1'(a) + \frac{P}{12} \right)^2 + \frac{1}{3}\phi_1'''(a) = 2\phi_{1,2} + \frac{P^2}{144}
\end{equation}
Substitute with the definition of $\wp$:
\begin{equation}
\wp^2(a) - \frac{1}{6}\wp''(a) = 2\phi_{1,2} + \frac{P^2}{144} = \frac{Q}{144}
\end{equation}


\begin{equation*}
(12\wp'(a))(\wp^2(a) - \frac{1}{6}\wp''(a) ) = (\frac{Q}{144})(12\wp'(a))
\end{equation*}
Integrate:
\begin{equation*}
[\wp'(a)]^2 = 4\wp^3(a) - \frac{Q}{12}\wp(a) + k
\end{equation*}
where $k= -\frac{R}{216}$ can be decided by Taylor expansions.

This is the \textbf{equation of elliptic functions} , the origin of modern algebraic geometry.

\begin{equation}
(\wp'(a))^2 = 4\wp^3(a) - \frac{Q}{12}\wp(a) - \frac{Q}{216}
\end{equation}
Usually we put $g_2 = \frac{Q}{12}, g_3 = \frac{R}{216}$.
\begin{Rmk}
 Assuming $\Im\tau>0$
\begin{equation}
\wp(\theta) = \frac{1}{\theta^2} + \sum\limits_{m,n}' \left( \frac{1}{(\theta - 2\pi n -2\pi\tau m)} - \frac{1}{(2\pi n + 2\pi \tau m)^2} \right)
\end{equation}
\end{Rmk}

\begin{Thm}[Ramanujan's DE]
\begin{align*}
 q \frac{dP}{dq} &= \frac{P^2-Q}{12}\\
 q \frac{dQ}{dq} &=\frac{PQ-R}{3}\\
 q \frac{dR}{dq} &= \frac{PR-Q^2}{2}
\end{align*}
\end{Thm}

This DE wasn't first discovered by Ramanujan. Actually, Halphen and Darboux had studied one of its equivance before Ramanujan.

Now let's prove Ramanujan's DE:

\begin{Def}
\begin{equation}
S_r = -\frac{B_{r+1}}{2(r+1)} + \phi_{0,r}
\end{equation}
\end{Def}

\begin{Eg}
  
\begin{itemize}
\item $S_1 = -\frac{B_2}{4} + \phi_{0,1} = - \frac{1}{24} + \sum\limits_{j=1}^{\infty}\sum\limits_{k=1}^{\infty} k q^{jk} = - \frac{P}{24}$.
\item $S_4 = \frac{Q}{240}$
\item $S_5 = - \frac{R}{504}$
\end{itemize}
\end{Eg}

\begin{equation}
\phi_1(\theta) = \frac{1}{2\theta} + \sum\limits_{n=1}^{\infty} \frac{(-1)^{n-1}}{(2n-1)!} S_{2n-1}\theta^{2n-1}
\end{equation}

\begin{equation}
\phi_2(\theta) = - \frac{1}{24} + \sum\limits_{n=1}^{\infty} \frac{(-1)^n}{(2n)!}\phi_{1,2n}\theta^{2n}
\end{equation}
\begin{equation}
\wp(\theta) = \frac{1}{\theta^2} + 2 \sum\limits_{n=1}^{\infty} \frac{(-1)^{n+1}}{(2n)!}S_{2n+1}\theta^{2n}
\end{equation}

From $\wp''(\theta) = 6\wp^2(\theta) - \frac{Q}{24}$, we learn that\footnote{To be calculated}
%To be calculated
\begin{equation}
S_{2n+3} = \frac{12(n+1)(2n+1)}{(n-1)(2n+5)} \sum\limits_{j=1}^{n-1} \binom{2n}{2j}S_{2j+1}S_{2n+1-2j}
\end{equation}
This implies that for $r\geq 1$, $S_{2r+1}\in \bm{Q}[Q,R]$ the ring of modular forms. Notice. that $S_1$ is not included here.

Actually, $S_{2n-1} = -\frac{B_{2n}}{4n} + \phi_{0,2n-1} = \sum\limits_{2i+3j=n}^{}K_{ij}Q^iR^j \quad K_{ij}\in \bm{Q}$

$\phi_1^2(\theta) + \frac{1}{2}\phi_1'(\theta) = \phi_2(\theta) + \frac{1}{2}\phi_2(0)$, so
\begin{equation}
\phi_1^2(\theta)+ \frac{1}{2}\phi'_1(\theta) = \phi_2(\theta) + \frac{S_1}{2}
\end{equation}
Again we expand them:\footnote{Some calculation is to be pasted here}
\begin{align*}
 LHS = &(\frac{1}{2\theta} + \sum\limits_{a=1}^{} \frac{(-1)^{a-1}}{(2a-1)!}S_{2a-1}\theta^{2a-1})(\frac{1}{2\theta} + \sum\limits_{b=1}^{} \frac{(-1)^{b-1}}{(2b-1)!} S_{2b-1}\theta^{2b-1})\\
       &+ \frac{1}{2} \left[ \frac{1}{2} + \sum\limits_{m=1}^{\infty} \frac{(-1)^{m-1}}{(2m-1)!}S_{2m-1}\theta^{2m-1} \right]\\
     = & \frac{1}{4\theta^2} + \sum\limits_{a=1}^{\infty} \frac{(-1)^{a-1}}{(2a-1)!}S_{2a-1}\theta^{2a-2} \\
       &+ \sum\limits_{a,b=1}^{}\frac{(-1)^{a+b-2}}{(2a-1)!(2b-1)!}S_{2a-1}S_{2b-1}\theta^{2a+2b-2}\\ %To be completed
RHS  =  & \frac{1}{4\theta^2} + \sum\limits_{m=1}^{\infty} \frac{(-1)^{m-1}}{(2m-2)!} S_{2m-1}\theta^{2m-2}
\end{align*}

So
\begin{equation}
\phi_{1,2n} = \frac{2n+3}{2(2n+1)} S_{2n+1} - \sum\limits_{j=1}^n \binom{2n}{2j-1}S_{2j-1}S_{2n+1-2j}
\end{equation}
Note that $S_1$ shows up here.
\begin{Eg}
  
\begin{itemize}
\item $\phi_{1,2} = \frac{5}{6}S_3 - 2S_1^2$
\item $\phi_{1,4} = \frac{7}{10}S_5 - 8 S_1S_3$
\item $\phi_{1,6} = \frac{400}{7}S_3^2 - 12 S_1S_5$
\end{itemize}
\end{Eg}

This also implies that $\phi_{1,2n} = \sum\limits_{i+2j+3l =n+1}^{}K_{ijk}P^iQ^jR^k$, which was first discovered by Ramanujan. So $\phi_{1,2n}\in \bm{Q}[P,Q,R]$, the ring of quasi-modular forms.

Substituting with $P,Q,R$ and we get

\begin{align*}
 288 \phi_{1,2} = & Q - P^2\\
 720\phi_{1,4} = & PQ-R\\
 1008 \phi_{1,6} = & Q^2 - PR
\end{align*}

$q \frac{\mathrm{d}}{\mathrm{d}q}\phi_{m,n} = \phi_{m+1,n+1}$, so $Q-P^2 = 288 q \frac{\mathrm{d}}{\mathrm{d}q}\phi_{0,1}, PQ-R = 720 q \frac{\mathrm{d}}{\mathrm{d}q}\phi_{0,3}, Q^2-PR=1008 q \frac{\mathrm{d}}{\mathrm{d}q}\phi_{0,5}$, which is exactly the Ramanujan's DE.

\paragraph{Applications}
The determinant of the polynomial $a(t-r_1)\cdots (t-r_n)$ is defined as $a\prod\limits_{1\leq i<j\leq n}(r_i-r_j)^2$. Consider the polynomial $4t^3 - \frac{Q}{12}t - \frac{R}{216}$, its discriminant is $Q^3-R^2$.

\begin{equation}
  \label{eq:1.app}
q \frac{\mathrm{d}}{\mathrm{d}q}\log(Q^3-R^2) = \frac{1}{Q^3-R^2} \left( 3Q^2 (\frac{PQ-R}{3}) - 2R (\frac{PR-Q^2}{2}) \right) = P
\end{equation}


Integrate \ref{eq:1.app}
\begin{equation}
\log(Q^3-R^2) = \log(q) + 24 \sum\limits_{n=1}^{\infty} \log(1-q^n) +c
\end{equation}
So $Q^3-R^2 = 1728q \prod\limits_{n=1}^{\infty} (1-q^n)^{24}$ (the coefficient is calculated via Taylor expansion), which is \ref{eq:0.id}.

To further study \ref{eq:0.id}, Ramanujan introduced the function $\tau(n)$ defined as
\begin{equation}
\tau(n)=q\prod_{n=1}^{\infty} (1-q^n)^{24}
\end{equation}
 $\tau(n) \equiv \sigma_{11}(n) \pmod{691}$. Here $\sigma_{r}(n) = \sum\limits_{d|n}^{}d^r$.\footnote{HW2.1:read corresponding parts of the textbook for details.}

\section{Jacobi's triple product identity}
Let $|q|<1$, $z\neq 0$. Define
\begin{equation}
f(z) = \prod_{n=1}^{\infty} (1+zq^{2n-1})(1+z^{-1}q^{2n-1})
\end{equation}
$f(z)$ converges absolutely and uniformly on compact subsets of the region $0<|z|<\infty$.

Denote its Laurant expansion:$f(z) = \sum\limits_{n\in \bm{Z}}^{} c_nz^n \quad 0<|z|<\infty$.

$f(z)$ is obviously symmetric $f(z)=f(z^{-1})$.

Note that
\begin{equation}
\frac{f(z)}{f(q^2z)} = \frac{1+zq}{1+z^{-1}q^{-1}} = zq
\end{equation}
In other words, \[qz f(q^2z) =f(z)\] is a functional equation satisfied by $f(z)$. Apply Laurant expansion and we get \[c_{n-1}q^{2n-1} = c_n\]. Thus
\begin{equation}
c_n=q^{n^2}c_0
\end{equation}
So
\begin{equation}
f(z)=c_0 \sum\limits_{n\in \bm{Z}}^{}q^{n^2}z^n
\end{equation}

So the problem reduces to calculating $c_0$.

\begin{Def}
 Assume $\alpha\beta\neq 0, |\alpha\beta q^2|<1$.
\begin{equation}
 \Phi(z,\alpha,\beta) =\prod_{n=1}^{\infty} \frac{(1+zq^{2n-1})(1+z^{-1}q^{2n-1})}{(1+\alpha zq^{2n-1})(1+\beta z^{-1}q^{2n-1})}
\end{equation}
\end{Def}

$\Phi(z,\alpha,\beta)$ is analytic in the region $|\beta q|<|z|<|\alpha q|^{-1}$, then
\begin{equation}
\Phi(z,\alpha,\beta) = \sum\limits_{n\in \bm{Z}}^{}  C_n(\alpha,\beta)z^n
\end{equation}



\end{document}
%%% Local Variables:
%%% mode: latex
%%% TeX-master: t
%%% End:
