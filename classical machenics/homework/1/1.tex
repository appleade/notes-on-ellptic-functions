\documentclass{ctexart}
\usepackage{amsmath,amssymb,amsthm,bm,ulem}
\usepackage[margin=1 in]{geometry}
\title{分析力学第1次作业}
\author{数91\and 董浚哲}
\date{\today}

\newcommand{\A}{\mathbb{A}}
\newcommand{\R}{\mathbb{R}}
\newcommand{\N}{\mathbb{N}}
\newcommand{\dd}{\,\mathrm{d}}
\newcommand{\st}{\text{ s.t. }}
\newcommand{\pp}[2]{\frcac{\partial #1}{\partial #2}}
\newcommand{\dif}[2]{\frac{\mathrm{d}#1}{\mathrm{d}#2}}
\newcommand{\nm}[1]{\left\|#1\right\|}
\newcommand{\dual}[1]{\left<#1\right>}
\newcommand{\wto}{\rightharpoonup}
\newcommand{\wsto}{\stackrel{*}{\rightharpoonup}}
\newcommand{\cvin}{\text{ in }}
\newcommand{\alev}{\text{ a.e. }}

\begin{document}
\maketitle

\section*{1.}
\paragraph{(i)}
\begin{proof}
  注意到$d(a,b),d(b,c)\in V$,即在线性空间$V$中,故由平行四边形法则:
  \[LHS=(a-b)+(b-c)=(a-c)=RHS\]
\end{proof}

\paragraph{(ii)}
\begin{proof}
  \textbf{单射:}设$\exists b_{1}\neq b_{2}\in V\st d_{a}(b_{1})=d_{a}(b_{2})$,则$0=d_{a}(b_{1})-d_{a}(b_{2})=(b_{1}-a)-(b_{2}-a)=(b_{1}-b_{2})\neq 0$,矛盾!

  \textbf{满射:}$\forall c\in V, c=(a+c)-a=d_{a}(a+c)$,故为满射。

  综上,$d_{a}$是双射。
\end{proof}








\end{document}

%%% Local Variables:
%%% mode: latex
%%% TeX-master: t
%%% End:
